\documentclass[11pt,oneside]{article}

	% History ================================================================
	% 2023.06.03 - Modified from Chase Murray's version
	% ========================================================================

    % STANDARD PACKAGES ======================================================
    \usepackage{datetime}
    \usepackage{graphicx}
    % \usepackage{ctex} % Allow Chinese characters
    \usepackage[utf8]{inputenc}
    \usepackage[american]{babel}
    \usepackage{amssymb}
    \usepackage[intlimits]{amsmath}
    \usepackage{amsfonts}
    \usepackage{amsthm}
    \usepackage{array}
    \usepackage{mdwlist}        
    \usepackage[labelsep=quad,indention=10pt]{subfig}
    \usepackage{algorithm}
    \usepackage[noend]{algpseudocode}
    \usepackage{lscape}
    \usepackage{rotating} % Allows \begin{sideways} \end{sideways} for vertical table headers.    
    \usepackage{threeparttable} % Allow footnotes in tables.    
    \usepackage{tabularx}
    \usepackage{multirow} % Allow table cells to span multiple rows/cols.
    \usepackage{makecell}
    \usepackage{longtable}
    \usepackage{url} % Allow \url{} and \href{url}{name}
    \usepackage{verbatim}    
    \usepackage{enumerate} % http://www.tex.ac.uk/cgi-bin/texfaq2html?label=enumerate
    \usepackage{color} % Allow colored fonts
    \usepackage[toc,page]{appendix}

    \usepackage{bm}

    \usepackage{tikz}
    \usepackage{diagbox}
    \usepackage{lastpage} % \pageref{LastPage} = total number of pages.
    \usepackage{ifthen}        
    \usepackage{setspace} % Allows \singlespacing, \onehalfspacing, \doublespacing 
    \usepackage{listings} % Allows formatting of Python code (and other languages)
    \usepackage{wrapfig}
    \usepackage[normalem]{ulem} % Allows strikethrough (\sout{text to strike})
    % \usepackage{subfigure}        % Allows subfigs/subfloats


    \usepackage{xcolor,colortbl}    % http://ctan.org/pkg/xcolor
    % \usepackage[table]{xcolor}    % https://tex.stackexchange.com/questions/50349/color-only-a-cell-of-a-table
    
    % Make sure that {color} and {xcolor} are called before mdframed
    \usepackage[framemethod=TikZ]{mdframed}    % Allows colored textbox

    \usepackage{lipsum}                     % Dummy text
    % ========================================================================

    % DEFINE PROGRAMMING FORMAT ++++++++++++++++++++++++++++++++++++++++++++++
        \lstset{language=Python}          % Set your language (you can change the language for each code-block optionally)

        \definecolor{mygreen}{rgb}{0,0.6,0}
        \definecolor{mygray}{rgb}{0.5,0.5,0.5}
        \definecolor{mymauve}{rgb}{0.58,0,0.82}

        \lstset{
          backgroundcolor=\color{gray!05!white},   % choose the background color; you must add \usepackage{color} or \usepackage{xcolor}; should come as last argument
          basicstyle=\ttfamily,                    % the size of the fonts that are used for the code
          breakatwhitespace=false,                 % sets if automatic breaks should only happen at whitespace
          breaklines=true,                         % sets automatic line breaking
          captionpos=t,                            % sets the caption-position to bottom
          commentstyle=\color{black},              % comment style
          deletekeywords={...},                    % if you want to delete keywords from the given language
          escapeinside={\%*}{*)},                  % if you want to add LaTeX within your code
          extendedchars=true,                      % lets you use non-ASCII characters; for 8-bits encodings only, does not work with UTF-8
          frame=single,                               % adds a frame around the code
          keepspaces=true,                         % keeps spaces in text, useful for keeping indentation of code (possibly needs columns=flexible)
          % keywordstyle=\color{blue},             % keyword style
          language=Python,                         % the language of the code
          morekeywords={*,...},                    % if you want to add more keywords to the set
          numbers=left,                            % where to put the line-numbers; possible values are (none, left, right)
          numbersep=5pt,                           % how far the line-numbers are from the code
          % numberstyle=\tiny\color{mygray},       % the style that is used for the line-numbers
          rulecolor=\color{black},                 % if not set, the frame-color may be changed on line-breaks within not-black text (e.g. comments (green here))
          showspaces=false,                        % show spaces everywhere adding particular underscores; it overrides 'showstringspaces'
          showstringspaces=false,                  % underline spaces within strings only
          showtabs=false,                          % show tabs within strings adding particular underscores
          stepnumber=1,                            % the step between two line-numbers. If it's 1, each line will be numbered
          % stringstyle=\color{mymauve},           % string literal style
          tabsize=4,                               % sets default tabsize to 2 spaces
          % title=\lstname,                        % show the filename of files included with \lstinputlisting; also try caption instead of title
          xleftmargin=35pt,
          xrightmargin=15pt, 
          aboveskip=0pt,
          belowskip=5pt
        }
    % ++++++++++++++++++++++++++++++++++++++++++++++++++++++++++++++++++++++++

    % DEFINE/RENEW SOME ENVIRONMENTS =========================================    
        \renewenvironment{abstract}
          {\normalfont\footnotesize
            \list{}{\labelwidth0pt
              \leftmargin20pt \rightmargin\leftmargin
              \listparindent\parindent \itemindent0pt
              \parsep0pt
              \let\fullwidthdisplay\relax
            }
            \item[\hskip\labelsep\bfseries\abstractname:] %
        }{
          \endlist}

        \newcommand{\keywordsname}{Keywords}
        \newenvironment{keywords}
          {\normalfont\footnotesize
            \list{}{\labelwidth0pt
              \leftmargin20pt \rightmargin\leftmargin
              \listparindent\parindent \itemindent0pt
              \parsep0pt
              \let\fullwidthdisplay\relax}
            \item[\hskip\labelsep\bfseries\keywordsname:]}{\endlist}

        \newcommand{\dochistname}{History}
        \newenvironment{DocHistory}
          {\normalfont\footnotesize
            \list{}{\labelwidth0pt
              \leftmargin20pt \rightmargin\leftmargin
              \listparindent\parindent \itemindent0pt
              \parsep0pt
              \let\fullwidthdisplay\relax}
            \item[\hskip\labelsep\bfseries\dochistname:]}{\endlist}
    % ========================================================================    

    % DEFINE PAGE FORMATTING +++++++++++++++++++++++++++++++++++++++++++++++++
        % Select Line Spacing:
        \singlespacing
        % \onehalfspacing        
        % \doublespacing    

        % Margins:
        \usepackage[letterpaper,left=1.0in,top=1.0in,right=1.0in,bottom=1.0in]{geometry}
    
        % Page Style
        \pagestyle{plain}    % Includes page number
        %\pagestyle{empty}    % Completely blank                

        % By default all math is set to inline mode. The \displaystyle command
        % ensures that we don't get small fractions or summations with limits
        % on the sides.
        \everymath{\displaystyle}    
        
        % http://tex.stackexchange.com/questions/5223/command-for-argmin-or-argmax
        \DeclareMathOperator*{\argmin}{arg\,min}

        % Allow flalign items to be split over multiple pages:
        \allowdisplaybreaks[1]   % See ftp://ftp.ams.org/pub/tex/doc/amsmath/amsldoc.pdf    
    % ++++++++++++++++++++++++++++++++++++++++++++++++++++++++++++++++++++++++

    % DEFINITION, THEOREM, AND LEMMA +++++++++++++++++++++++++++++++++++++++++

        \theoremstyle{definition}
            \newtheorem{definition}{Definition}[section]
            \newtheorem*{example}{Example}
            \newtheorem{problem}{Problem}[section]
            \newtheorem*{solution}{Solution}
            \newtheorem{hypothesis}{Hypothesis}[section]
        \theoremstyle{plain}
            \newtheorem{theorem}{Theorem}[section]
            \newtheorem{corollary}{Corollary}[theorem]
            \newtheorem{lemma}[theorem]{Lemma}
            \newtheorem{conjecture}{Conjecture}
            \newtheorem{proposition}{Proposition}
        \theoremstyle{remark}
            \newtheorem*{remark}{Remark}

    % ++++++++++++++++++++++++++++++++++++++++++++++++++++++++++++++++++++++++

    % CUSTOM MACROS ++++++++++++++++++++++++++++++++++++++++++++++++++++++++++

        % This is how you may create a new variable:
        % \newcommand{\docjunk}{ text to display }
        
        % See https://gist.github.com/benkehoe/c46647134d4bbd514869
        % for more examples.

        % Create a box marked ``To Do'' around text.
        % \todo{  insert text here  }.
        \newcommand{\todo}[1]{\vspace{5 mm}\par \noindent
        \marginpar{\textsc{to do}}
        \framebox{\begin{minipage}[c]{0.95 \textwidth}
        \tt\begin{center} #1 \end{center}\end{minipage}}\vspace{5 mm}\par}

        % Create an empty box marked ``Result'' in the margin.
        % Specify the number of empty rows.
        % \result{8 em}.
        \newcommand{\result}[1]{\vspace{5 mm}\par \noindent
        \marginpar{\textsc{Result}} $\qquad\qquad$
        \framebox{\begin{minipage}[c]{0.75 \textwidth}
        \tt\begin{center} \vspace{#1} \end{center}\end{minipage}}\vspace{5 mm}\par}

        % Color selected text in red font.
        % \alert{text to color}
        \newcommand{\alert}[1]{{\color{red}#1}}

        % Color selected text in blue font.
        % \edited{text to color}
        \newcommand{\edited}[1]{{\color{blue}#1}}

        % Color selected text and add a "FIXME" note in the margin.
        % \fixme{text to color}
        \newcommand{\fixme}[1]{{\color{red}#1}
            \marginpar{\textsc{\color{red}fixme}}}

        % Color selected text (optional) and add a note in brackets.
        % \note[selected text]{comments}
        % \note{comments}
        \renewcommand{\note}[2][]{
            {\color{blue}#1 %
            [\textsc{note}:~#2]}
        }
        
        % Color selected text (optional) and add a note from someone.
        % \notefrom[selected text]{from}{comments}
        % \notefrom{from}{comments}
        \newcommand{\notefrom}[3][]{
            {\color{green!50!black}#1 %
            [\textsc{from #2}:~#3]}
        }
        
        % Color selected text (optional) and add a note to someone.
        % \noteto[selected text]{to}{comments}
        % \noteto{to}{comments}
        \newcommand{\noteto}[3][]{
            {\color{red}#1 %
            [\textsc{to #2}:~#3]}
        }

        % Color and Line Settings for Boxed Text
        \mdfsetup{
        % middlelinecolor=red,
        middlelinewidth=1pt,
        % linecolor=blue,
        % linewidth=1pt,
        backgroundcolor=orange!10!white,
        linecolor=orange!50!black,
        roundcorner=5pt}
        
        % Shortcut for referencing figures/tables:
        % Usage:  \figref{fig:name} --> Figure 1.
        \newcommand{\figref}[1]{\figurename~\ref{#1}}
        \newcommand{\tabref}[1]{\tablename~\ref{#1}}
    % ++++++++++++++++++++++++++++++++++++++++++++++++++++++++++++++++++++++++

    % SETUP TikZ +++++++++++++++++++++++++++++++++++++++++++++++++++++++++++++
        \usetikzlibrary{arrows,shapes,matrix}
        \usetikzlibrary{decorations.pathmorphing} 
        \usepgflibrary{plotmarks}
        \usetikzlibrary{patterns}  
        \usetikzlibrary{positioning} 
        \usetikzlibrary{snakes}  
        \tikzstyle{block}=[draw opacity=0.7,line width=1.4cm]
        
        % MORE STUFF TO ADD HERE?
    % ++++++++++++++++++++++++++++++++++++++++++++++++++++++++++++++++++++++++

    % SETUP BIBLIOGRAPHY +++++++++++++++++++++++++++++++++++++++++++++++++++++
    % [This section MUST be used if you have a bibliography.    ]
    % [Otherwise, leave this section commented out.        ]
    % \begin{comment}

        % FIXME -- EXPLAIN
        
        % Setup the Bibliography Style -- Select ONE of the following:
        % \usepackage{natbib}
        % \usepackage[sectionbib,square]{natbib}     %%% See natbib.pdf for explanation.
        % \usepackage[sectionbib,round]{natbib}
        \usepackage[square,numbers]{natbib}

        \bibliographystyle{plainnat}

        % Natbib setup for author-year style
        % \bibpunct has 1 optional and 6 mandatory arguments:
        %  [0.] The character preceding a post-note, default is a comma plus space. In redefining this character, 
        %     one must include a space if one is wanted. 
        %  1. the opening bracket symbol, default = (
        %  2. the closing bracket symbol, default = )
        %  3. the punctuation between multiple citations, default = ;
        %  4. the letter `n' for numerical style, or `s' for numerical superscript style, 
        %    any other letter for author-year, default = author-year;
        %  5. the punctuation that comes between the author names and the year
        %  6. the punctuation that comes between years or numbers when common author lists are suppressed (default = ,);

        % Natbib setup for author-year style
        \bibpunct[, ]{(}{)}{,}{a}{}{,}                % Use author names
        % \bibpunct[, ]{[}{]}{,}{n}{}{,}            % Use numbers
        
        \def\bibfont{\small}
        \def\bibsep{\smallskipamount}
        \def\bibhang{24pt}
        \def\newblock{\ }
        \def\BIBand{and}
    % \end{comment}
    % ++++++++++++++++++++++++++++++++++++++++++++++++++++++++++++++++++++++++

    % DOCUMENT INFO ++++++++++++++++++++++++++++++++++++++++++++++++++++++++++
        \newcommand{\docTitle}{}

        % List authors here, separated by \and 
        \newcommand{\docAuthor}{}
        % \newcommand{\docAuthor}{}

        \newcommand{\docAffil}{
            School of Management, Shanghai University, Shanghai, China
        }

        \newcommand{\docAbstract}{}

        \newcommand{\docKeyword}{}

        % This date will appear under the title.
        \newcommand{\docDate}{\today}       % {} --> don't show a date.
            
        % This date will appear in the page header:
        \newcommand{\draftDate}{\today}    % {\today} --> draft, {} --> finalized (hidden)
    
        % The image files should be saved here:
        \graphicspath{ {../../image/} }
    % ++++++++++++++++++++++++++++++++++++++++++++++++++++++++++++++++++++++++

    % DEFINE HEADER ++++++++++++++++++++++++++++++++++++++++++++++++++++++++++
        \usepackage{fancyhdr}
        \pagestyle{fancy}
        \ifthenelse{\equal{\draftDate}{}}
            {
                % This is the final version...remove the date from the header
                \chead{}
            }
            {
                % This is a working draft...include the date in the header
                % \chead{\color{red}DRAFT -- Updated \draftDate~at~\currenttime}
            }
        \lhead{}    % no left/right header content
        \rhead{}
        %\cfoot{}
        %\lfoot{}
        %\rfoot{}
        \renewcommand{\headrulewidth}{0pt}
        \renewcommand{\footrulewidth}{0pt}
        %\fancyfoot{}
    % ++++++++++++++++++++++++++++++++++++++++++++++++++++++++++++++++++++++++
    
    % DEFINE PROGRAMMING FORMAT ++++++++++++++++++++++++++++++++++++++++++++++
    \lstset{language=Python}          % Set your language (you can change the language for each code-block optionally)

    \definecolor{mygreen}{rgb}{0,0.6,0}
    \definecolor{mygray}{rgb}{0.5,0.5,0.5}
    \definecolor{mymauve}{rgb}{0.58,0,0.82}

    \lstset{ %
      backgroundcolor=\color{gray!05!white},   % choose the background color; you must add \usepackage{color} or \usepackage{xcolor}; should come as last argument
      basicstyle=\ttfamily,        % the size of the fonts that are used for the code
      breakatwhitespace=false,         % sets if automatic breaks should only happen at whitespace
      breaklines=true,                 % sets automatic line breaking
      captionpos=t,                    % sets the caption-position to bottom
      commentstyle=\color{black},    % comment style
      deletekeywords={...},            % if you want to delete keywords from the given language
      escapeinside={\%*}{*)},          % if you want to add LaTeX within your code
      extendedchars=true,              % lets you use non-ASCII characters; for 8-bits encodings only, does not work with UTF-8
      frame=single,                       % adds a frame around the code
      keepspaces=true,                 % keeps spaces in text, useful for keeping indentation of code (possibly needs columns=flexible)
      % keywordstyle=\color{blue},       % keyword style
      language=Python,                 % the language of the code
      morekeywords={*,...},           % if you want to add more keywords to the set
      numbers=none,                    % where to put the line-numbers; possible values are (none, left, right)
      numbersep=5pt,                   % how far the line-numbers are from the code
      % numberstyle=\tiny\color{mygray}, % the style that is used for the line-numbers
      rulecolor=\color{black},         % if not set, the frame-color may be changed on line-breaks within not-black text (e.g. comments (green here))
      showspaces=false,                % show spaces everywhere adding particular underscores; it overrides 'showstringspaces'
      showstringspaces=false,          % underline spaces within strings only
      showtabs=false,                  % show tabs within strings adding particular underscores
      stepnumber=1,                    % the step between two line-numbers. If it's 1, each line will be numbered
      % stringstyle=\color{mymauve},     % string literal style
      tabsize=4,                       % sets default tabsize to 2 spaces
      % title=\lstname,                   % show the filename of files included with \lstinputlisting; also try caption instead of title
      xleftmargin=35pt,
      xrightmargin=15pt, 
      aboveskip=0pt,
      belowskip=5pt
    }
    % ++++++++++++++++++++++++++++++++++++++++++++++++++++++++++++++++++++++++

    \newcommand{\titleSec}{
        % See https://tex.stackexchange.com/questions/216098/redefine-maketitle
        \begin{center}
        % \let \footnote \thanks
        {\Large \textbf{\docTitle} \par}

        % Authors?
        % Comment these lines out if you want to hide authors
        \vskip 1.0em%
        \lineskip .5em%
        \begin{tabular}[t]{c}
            \docAuthor
        \end{tabular}\par%

        % Affiliation?
        % Comment these lines out if you want to hide affiliation info
        \vskip 1.0em%
        {\small \docAffil \par}

        % Displayed date?
        % Comment these lines out if you want to hide the date
        %\vskip 1.0em%
        %{\small \docDate \par}  

        \end{center}
        \par
        \vskip 1.5em

        % \begin{abstract}
        %     \docAbstract
        % \end{abstract}

        % \begin{keywords}
        %     \docKeyword
        % \end{keywords}

        % This is version \texttt{\templateVersion} of this template.
        % Visit \templatesURL for the latest versions.
    }
\usepackage{makecell}
\setlength{\parindent}{0pt}

\usetikzlibrary{shapes.geometric, arrows}
    \tikzstyle{startstop} = [rectangle, rounded corners, minimum width=3cm, minimum height=1cm,text centered, draw=black]
    \tikzstyle{io} = [trapezium, trapezium left angle=70, trapezium right angle=110, minimum width=3cm, minimum height=1cm, text centered, draw=black]
    \tikzstyle{process} = [rectangle, minimum width=2cm, minimum height=1cm, text centered, draw=black, inner sep=0.1cm]
    \tikzstyle{decision} = [diamond, minimum width=2cm, minimum height=0cm, text centered, draw=black, inner sep=0cm]
    \tikzstyle{arrow} = [thick,->,>=stealth]
    \tikzstyle{branchnode} = [circle, minimum size = 1cm, text centered, draw=black, inner sep=0.1cm]

\renewcommand{\docTitle}{Special Topic - Formulation Tips}
\renewcommand{\docAuthor}{Lan Peng, Ph.D.}
\renewcommand{\docAffil}{School of Management, Shanghai University, Shanghai, China}
\begin{document}
    \titleSec

    \begin{center}
        \textit{``Trick or Treat.''}
    \end{center}

    \section{Linear Program Formulation Tips}
        \paragraph{Absolute Value}
            
            Consider the following model statement:
            \begin{align*}
                \min \quad & \sum_{j\in J}c_j|x_j|, \quad c_j > 0 \\
                \text{s.t.} \quad & \sum_{j\in J}a_{ij}x_j \gtreqless b_i, \quad \forall i\in I \\
                                  & x_j \quad \text{unrestricted}, \quad \forall j\in J 
            \end{align*}

            
            Equivalent model:
            \begin{align*}
                \min \quad & \sum_{j\in J}c_j(x_j^+ + x_j^-), \quad c_j > 0 \\
                \text{s.t.} \quad & \sum_{j\in J}a_{ij}(x_j^+ - x_j^-) \gtreqless b_i, \quad \forall i\in I \\
                                  & x_j^+, x_j^- \ge 0, \quad \forall j\in J 
            \end{align*}

        \paragraph{Minimax Objective}
            
            Consider the following model statement:
            \begin{align*}
                \min \quad & \max_{k\in K}\sum_{j\in J}c_{kj}x_j \\
                \text{s.t.} \quad & \sum_{j\in J}a_{ij}x_j \gtreqless b_i, \quad \forall i\in I \\
                                  & x_j \ge 0, \quad \forall j\in J 
            \end{align*}
            
            Equivalent model:
            \begin{align*}
                \min \quad & z \\
                \text{s.t.} \quad & \sum_{j\in J}a_{ij}x_j \gtreqless b_i, \quad \forall i\in I \\
                                  & \sum_{j\in J}c_{kj}x_j \le z, \quad \forall k\in K \\
                                  & x_j \ge 0, \quad \forall j\in J 
            \end{align*}
        \paragraph{Fractional Objective}
            
            Consider the following model statement:
            \begin{align*}
                \min \quad & \frac{\sum_{j\in J}c_{j}x_j + \alpha}{\sum_{j\in J}d_{j}x_j + \beta} \\
                \text{s.t.} \quad & \sum_{j\in J}a_{ij}x_j \gtreqless b_i, \quad \forall i\in I \\
                                  & x_j \ge 0, \quad \forall j\in J 
            \end{align*}
            
            Equivalent model:
            \begin{align*}
                \min \quad & \sum_{j\in J}c_{j}x_jt + \alpha t \\
                \text{s.t.} \quad & \sum_{j\in J}a_{ij}x_j \gtreqless b_i, \quad \forall i\in I \\
                                  & \sum_{j\in J}d_jx_jt + \beta t = 1\\
                                  & t > 0 \\
                                  & x_j \ge 0, \quad \forall j\in J \\
                                  & (t = \frac1{\sum_{j\in J}d_jx_j + \beta})
            \end{align*}
        \paragraph{Range Constraint}
            
            Consider the following model statement:
            \begin{align*}
                \min \quad & \sum_{j\in J}c_jx_j \\
                \text{s.t.} \quad & d_i\le \sum_{j\in J}a_{ij}x_j \le e_i, \quad \forall i\in I \\
                                  & x_j \ge 0, \quad \forall j\in J 
            \end{align*}
            
            Equivalent model:
            \begin{align*}
                \min \quad & \sum_{j\in J}c_jx_j, \quad c_j > 0 \\
                \text{s.t.} \quad & u_i + \sum_{j\in J}a_{ij}x_j = e_i, \quad \forall i\in I \\
                                  & x_j \ge 0, \quad \forall j\in J \\
                                  & 0\le u_i \le e_i-d_i, \quad \forall i\in I 
            \end{align*}

    \section{Integer Program Formulation Tips}
        \paragraph{A Variable Taking Discontinuous Values}
             In algebraic notation: 
            \begin{equation*}
                x = 0,\quad \text{or} \quad l\le x \le u 
            \end{equation*}
            
            Equivalent model:
            \begin{align*}
                & x \le uy \\
                & x \ge ly  \\
                & y \in \{0, 1\} 
            \end{align*}
            where
            \begin{equation*}y=\begin{cases}0, & \text{if }x=0 \\ 1, & \text{if } l\le x \le u\end{cases} \end{equation*}
                
        \paragraph{Fixed Costs}
             In algebraic notation: 
            \begin{equation*}
                C(x) = \begin{cases} 0 & \text{for } x=0 \\ k + cx & \text{for } x > 0 \end{cases} 
            \end{equation*}
            
            Equivalent model:
            \begin{align*}
                & C^*(x, y) = ky+cx\\
                & x \le My  \\
                & x \ge 0 \\
                & y \in \{0, 1\} 
            \end{align*}
            where
            \begin{equation*}y=\begin{cases}0, & \text{if }x=0 \\ 1, & \text{if }x\ge 0\end{cases} \end{equation*}
        
        \paragraph{Either-or Constraints}
             In algebraic notation: 
            \begin{equation*}
                \sum_{j\in J} a_{1j} x_j \le b_1 \text{ or } \sum_{j\in J} a_{2j} x_j \le b_2 
            \end{equation*}
            
            Equivalent model:
            \begin{align*}
                & \sum_{j\in J} a_{1j} x_j \le b_1 + M_1y  \\
                & \sum_{j\in J} a_{2j} x_j \le b_2 + M_1(1-y)  \\
                & y \in \{0, 1\} 
            \end{align*}
            where
            \begin{equation*}y=\begin{cases}0, & \text{if }\sum_{j\in J} a_{1j} x_j \le b_1 \\ 1, & \text{if } \sum_{j\in J} a_{2j} x_j \le b_2\end{cases} \end{equation*}
            Notice that the sign before $M$ is determined by the inequality $\ge$ or $\le$, if it is \lq\lq{}$\ge$\rq\rq{}, use \lq\lq{}$-$\rq\rq{}, if it \lq\lq{}$\le$\rq\rq{}, use \lq\lq{}+\rq\rq{}.
        
        \paragraph{Conditional Constraints}
             If constraint A is satisfied, then constraint B must also be satisfied
            \begin{equation*}
                \text{If} \quad \sum_{j\in J} a_{1j} x_j \le b_1 \text{ then } \sum_{j\in J} a_{2j} x_j \le b_2 
            \end{equation*}
            The key part is to find the opposite of the first condition. We are using $A\Rightarrow B \Leftrightarrow \neg B \Rightarrow \neg A$\\
            Therefore it is equivalent to
            \begin{equation*}
                \sum_{j\in J} a_{1j} x_j > b_1 \text{ or } \sum_{j\in J} a_{2j} x_j \le b_2 
            \end{equation*}
            Furthermore, it is equivalent to
            \begin{equation*}
                \sum_{j\in J} a_{1j} x_j \ge b_1 + \epsilon \text{ or } \sum_{j\in J} a_{2j} x_j \le b_2 
            \end{equation*}
            Where $\epsilon$ is a very small positive number.\\
            
            Equivalent model:
            \begin{align*}
                & \sum_{j\in J} a_{1j} x_j \ge b_1 + \epsilon -  M_2y  \\
                & \sum_{j\in J} a_{2j} x_j \le b_2 + M_2(1-y)  \\
                & y \in \{0, 1\} 
            \end{align*} 
        
        \paragraph{Special Ordered Sets}
            SOS1 Description: Out of a set of yes-no decisions, at most one decision variable can be yes. \
            \begin{align*}
                x_1=1,x_2=x_3&=\dots=x_n=0  \\
                &\text{or}  \\
                x_2=1, x_1=x_3&=\dots=x_n=0  \\
                &\text{or ...} 
            \end{align*} 
            
            Equivalent model:
            \begin{equation*} \sum_{i} x_i = 1, \quad i \in N \end{equation*}
            
            SOS2 Description: Out of a set of binary variables, at most two variables can be nonzero. In addition, the two variables must be adjacent to each other in a fixed order list.\\
            
            Equivalent model:
            If $x_1, x_2, ... , x_n$ is a SOS2, then
            \begin{align*}
                & \sum_{i=1}^{n} x_i \le 2  \\
                & x_i + x_j \le 1, \forall i \in \{1, 2,..., n\}, j \in \{i+2, i+3, ..., n\}  \\
                &x_i \in \{0, 1\}
            \end{align*}
                            
        \paragraph{Piecewise Linear Formulations}
             The objective function is a sequence of line segments, e.g. $y=f(x), $ consists $k-1$ linear segments going through $k$ given points $(x_1, y_1), (x_2, y_2), ... ,(x_k, y_k)$.\\
            Denote 
            \begin{equation*}d_i=\begin{cases}1, & x\in (x_i, x_{i+1})\\0, & \text{otherwise} \end{cases}\end{equation*}
            Then the objective function is
            \begin{equation*}\sum_{i \in \{1, 2, ..., k-1\}} y = d_if_i(x) \end{equation*} 
            
            Equivalent model: Given that objective function as a piecewise linear formulation, we can have these constraints\\
            \begin{align*}
                &\sum_{i \in \{1, 2, ..., k-1\}} d_i =1  \\
                &d_i \in \{0, 1\}, i \in \{1, 2, ..., k-1\}  \\
                & x = \sum_{i \in \{1, 2, ..., k\}} w_i x_i  \\
                & y = \sum_{i \in \{1, 2, ..., k\}} w_i y_i  \\
                & w_1 \le d_1  \\
                & w_i \le d_{i-1} + d{i}, i \in \{2, 3, ..., k-1\}  \\
                & w_k \le d_{k-1} 
            \end{align*}
            In this case, $ w_i \in SOS2$ (second definition)       
                                
        \paragraph{Conditional Binary Variables}
             Choose at most $n$ binary variable to be 1 out of  $x_1, x_2, ... x_m, m\ge n$\\
            If $n=1$ then it is SOS1.\\
            
            Equivalent model:
            \begin{equation*}
                \sum_{k\in \{1,2,...,m\}} x_k \le n
            \end{equation*}
             Choose exactly $n$ binary variable to be 1 out of  $x_1, x_2, ... x_m, m\ge n$\\
            
            Equivalent model:
            \begin{equation*}
                \sum_{k\in \{1,2,...,m\}} x_k = n
            \end{equation*}
             Choose $x_j$ only if $x_k = 1$\\
            
            Equivalent model:
            \begin{equation*}x_j = x_k  \end{equation*}
             \lq\lq{}and\rq\rq{} condition, iff $x_1, x_2, ... , x_m =1$ then $y=1$\\
            
            Equivalent model:
            \begin{align*}
                & y \le x_i, i\in \{1, 2, ..., m\}  \\
                & y \ge \sum_{i \in \{1, 2, ..., m\}} x_i - (m - 1) 
            \end{align*}

        \paragraph{Elimination of Products of Variables}
             For variables $x_1$ and $x_2$,
            \begin{equation*}y = x_1 x_2\end{equation*}
            
            Equivalent model: If $x_1, x_2$ are binary, it is the same as \lq\lq{}and\rq\rq{} condition of binary variables.\\
            If $x_1$ is binary, while $x_2$ is continuous and $0 \le x_2 \le u$, then
            \begin{align*}
                y &\le ux_1  \\
                y &\le x_2  \\
                y &\ge x_2 - u(1- x_1)  \\
                y &\ge 0 
            \end{align*}
            If both $x_1$ and $x_2$ are continuous, it is non-linear, we can use Piecewise linear formulation to simulate.

    % \bibliography{literature}
\end{document}