\documentclass[12pt,oneside]{book}

	% History ================================================================
	% 2023.06.03 - Modified from Chase Murray's version
	% ========================================================================

    % STANDARD PACKAGES ======================================================
    \usepackage{datetime}
    \usepackage{graphicx}
    % \usepackage{ctex} % Allow Chinese characters
    \usepackage[utf8]{inputenc}
    \usepackage[american]{babel}
    \usepackage{amssymb}
    \usepackage[intlimits]{amsmath}
    \usepackage{amsfonts}
    \usepackage{amsthm}
    \usepackage{array}
    \usepackage{mdwlist}        
    % \usepackage[labelsep=quad,indention=10pt]{subfig}
    \usepackage{algorithm}
    \usepackage[noend]{algpseudocode}
    \usepackage{lscape}
    \usepackage{rotating} % Allows \begin{sideways} \end{sideways} for vertical table headers.    
    \usepackage{threeparttable} % Allow footnotes in tables.    
    \usepackage{tabularx}
    \usepackage{multirow} % Allow table cells to span multiple rows/cols.
    \usepackage{makecell}
    \usepackage{longtable}
    \usepackage{url} % Allow \url{} and \href{url}{name}
    \usepackage{verbatim}    
    \usepackage{enumerate} % http://www.tex.ac.uk/cgi-bin/texfaq2html?label=enumerate
    \usepackage{color} % Allow colored fonts
    \usepackage[toc,page]{appendix}

    \usepackage{bm}

    \usepackage{tikz}
        \usetikzlibrary{shapes.geometric, arrows}
            \tikzstyle{startstop} = [rectangle, rounded corners, minimum width=3cm, minimum height=1cm,text centered, draw=black]
            \tikzstyle{io} = [trapezium, trapezium left angle=70, trapezium right angle=110, minimum width=3cm, minimum height=1cm, text centered, draw=black]
            \tikzstyle{process} = [rectangle, minimum width=2cm, minimum height=1cm, text centered, draw=black, inner sep=0.1cm]
            \tikzstyle{decision} = [diamond, minimum width=2cm, minimum height=0cm, text centered, draw=black, inner sep=0cm]
            \tikzstyle{arrow} = [thick,->,>=stealth]
            \tikzstyle{branchnode} = [circle, minimum size = 1cm, text centered, draw=black, inner sep=0.1cm]
            \tikzstyle{solidNode} = [circle, minimum size = 0.1cm, fill=black]
            \tikzstyle{link} = [thick, -]
            \tikzstyle{matchedLink} = [decorate, decoration={snake}]
            \tikzstyle{circleNode} = [
                circle, 
                minimum size = 0.7cm, 
                text centered, 
                draw=black, 
                inner sep=0.1cm
            ]
    \usepackage{diagbox}
    \usepackage{lastpage} % \pageref{LastPage} = total number of pages.
    \usepackage{ifthen}        
    \usepackage{setspace} % Allows \singlespacing, \onehalfspacing, \doublespacing 
    \usepackage{listings} % Allows formatting of Python code (and other languages)
    % \usepackage{wrapfig}
    \usepackage[normalem]{ulem} % Allows strikethrough (\sout{text to strike})
    % \usepackage{subfigure}        % Allows subfigs/subfloats


    \usepackage{xcolor,colortbl}    % http://ctan.org/pkg/xcolor
    % \usepackage[table]{xcolor}    % https://tex.stackexchange.com/questions/50349/color-only-a-cell-of-a-table
    
    % Make sure that {color} and {xcolor} are called before mdframed
    \usepackage[framemethod=TikZ]{mdframed}    % Allows colored textbox

    \usepackage{lipsum}                     % Dummy text
    % ========================================================================

    % DEFINE PROGRAMMING FORMAT ++++++++++++++++++++++++++++++++++++++++++++++
        \lstset{language=Python}          % Set your language (you can change the language for each code-block optionally)

        \definecolor{mygreen}{rgb}{0,0.6,0}
        \definecolor{mygray}{rgb}{0.5,0.5,0.5}
        \definecolor{mymauve}{rgb}{0.58,0,0.82}

        \lstset{
          backgroundcolor=\color{gray!05!white},   % choose the background color; you must add \usepackage{color} or \usepackage{xcolor}; should come as last argument
          basicstyle=\ttfamily,                    % the size of the fonts that are used for the code
          breakatwhitespace=false,                 % sets if automatic breaks should only happen at whitespace
          breaklines=true,                         % sets automatic line breaking
          captionpos=t,                            % sets the caption-position to bottom
          commentstyle=\color{black},              % comment style
          deletekeywords={...},                    % if you want to delete keywords from the given language
          escapeinside={\%*}{*)},                  % if you want to add LaTeX within your code
          extendedchars=true,                      % lets you use non-ASCII characters; for 8-bits encodings only, does not work with UTF-8
          frame=single,                               % adds a frame around the code
          keepspaces=true,                         % keeps spaces in text, useful for keeping indentation of code (possibly needs columns=flexible)
          % keywordstyle=\color{blue},             % keyword style
          language=Python,                         % the language of the code
          morekeywords={*,...},                    % if you want to add more keywords to the set
          numbers=left,                            % where to put the line-numbers; possible values are (none, left, right)
          numbersep=5pt,                           % how far the line-numbers are from the code
          % numberstyle=\tiny\color{mygray},       % the style that is used for the line-numbers
          rulecolor=\color{black},                 % if not set, the frame-color may be changed on line-breaks within not-black text (e.g. comments (green here))
          showspaces=false,                        % show spaces everywhere adding particular underscores; it overrides 'showstringspaces'
          showstringspaces=false,                  % underline spaces within strings only
          showtabs=false,                          % show tabs within strings adding particular underscores
          stepnumber=1,                            % the step between two line-numbers. If it's 1, each line will be numbered
          % stringstyle=\color{mymauve},           % string literal style
          tabsize=4,                               % sets default tabsize to 2 spaces
          % title=\lstname,                        % show the filename of files included with \lstinputlisting; also try caption instead of title
          xleftmargin=35pt,
          xrightmargin=15pt, 
          aboveskip=0pt,
          belowskip=5pt
        }
    % ++++++++++++++++++++++++++++++++++++++++++++++++++++++++++++++++++++++++

    % DEFINE/RENEW SOME ENVIRONMENTS =========================================    
        % \renewenvironment{abstract}
        %   {\normalfont\footnotesize
        %     \list{}{\labelwidth0pt
        %       \leftmargin20pt \rightmargin\leftmargin
        %       \listparindent\parindent \itemindent0pt
        %       \parsep0pt
        %       \let\fullwidthdisplay\relax
        %     }
        %     \item[\hskip\labelsep\bfseries\abstractname:] %
        % }{
        %   \endlist}

        % \newcommand{\keywordsname}{Keywords}
        % \newenvironment{keywords}
        %   {\normalfont\footnotesize
        %     \list{}{\labelwidth0pt
        %       \leftmargin20pt \rightmargin\leftmargin
        %       \listparindent\parindent \itemindent0pt
        %       \parsep0pt
        %       \let\fullwidthdisplay\relax}
        %     \item[\hskip\labelsep\bfseries\keywordsname:]}{\endlist}

        % \newcommand{\dochistname}{History}
        % \newenvironment{DocHistory}
        %   {\normalfont\footnotesize
        %     \list{}{\labelwidth0pt
        %       \leftmargin20pt \rightmargin\leftmargin
        %       \listparindent\parindent \itemindent0pt
        %       \parsep0pt
        %       \let\fullwidthdisplay\relax}
        %     \item[\hskip\labelsep\bfseries\dochistname:]}{\endlist}
    % ========================================================================    

    % DEFINE PAGE FORMATTING +++++++++++++++++++++++++++++++++++++++++++++++++
        % Select Line Spacing:
        \singlespacing
        % \onehalfspacing        
        % \doublespacing    

        % Margins:
        \usepackage[letterpaper,left=1.0in,top=1.0in,right=1.0in,bottom=1.0in]{geometry}
    
        % Page Style
        \pagestyle{plain}    % Includes page number
        %\pagestyle{empty}    % Completely blank                

        % By default all math is set to inline mode. The \displaystyle command
        % ensures that we don't get small fractions or summations with limits
        % on the sides.
        \everymath{\displaystyle}    
        
        % http://tex.stackexchange.com/questions/5223/command-for-argmin-or-argmax
        \DeclareMathOperator*{\argmin}{arg\,min}

        % Allow flalign items to be split over multiple pages:
        \allowdisplaybreaks[1]   % See ftp://ftp.ams.org/pub/tex/doc/amsmath/amsldoc.pdf    
    % ++++++++++++++++++++++++++++++++++++++++++++++++++++++++++++++++++++++++

    % DEFINITION, THEOREM, AND LEMMA +++++++++++++++++++++++++++++++++++++++++

        \theoremstyle{definition}
            \newtheorem{definition}{Definition}[section]
            \newtheorem*{example}{Example}
            \newtheorem{problem}{Problem}[section]
            \newtheorem*{solution}{Solution}
            \newtheorem{hypothesis}{Hypothesis}[section]
        \theoremstyle{plain}
            \newtheorem{theorem}{Theorem}[section]
            \newtheorem{corollary}{Corollary}[theorem]
            \newtheorem{lemma}[theorem]{Lemma}
            \newtheorem{conjecture}{Conjecture}
            \newtheorem{proposition}{Proposition}
        \theoremstyle{remark}
            \newtheorem*{remark}{Remark}

    % ++++++++++++++++++++++++++++++++++++++++++++++++++++++++++++++++++++++++

    % CUSTOM MACROS ++++++++++++++++++++++++++++++++++++++++++++++++++++++++++

        % This is how you may create a new variable:
        % \newcommand{\docjunk}{ text to display }
        
        % See https://gist.github.com/benkehoe/c46647134d4bbd514869
        % for more examples.

        % Create a box marked ``To Do'' around text.
        % \todo{  insert text here  }.
        \newcommand{\todo}[1]{\vspace{5 mm}\par \noindent
        \marginpar{\textsc{to do}}
        \framebox{\begin{minipage}[c]{0.95 \textwidth}
        \tt\begin{center} #1 \end{center}\end{minipage}}\vspace{5 mm}\par}

        % Create an empty box marked ``Result'' in the margin.
        % Specify the number of empty rows.
        % \result{8 em}.
        \newcommand{\result}[1]{\vspace{5 mm}\par \noindent
        \marginpar{\textsc{Result}} $\qquad\qquad$
        \framebox{\begin{minipage}[c]{0.75 \textwidth}
        \tt\begin{center} \vspace{#1} \end{center}\end{minipage}}\vspace{5 mm}\par}

        % Color selected text in red font.
        % \alert{text to color}
        \newcommand{\alert}[1]{{\color{red}#1}}

        % Color selected text in blue font.
        % \edited{text to color}
        \newcommand{\edited}[1]{{\color{blue}#1}}

        % Color selected text and add a "FIXME" note in the margin.
        % \fixme{text to color}
        \newcommand{\fixme}[1]{{\color{red}#1}
            \marginpar{\textsc{\color{red}fixme}}}

        % Color selected text (optional) and add a note in brackets.
        % \note[selected text]{comments}
        % \note{comments}
        \renewcommand{\note}[2][]{
            {\color{blue}#1 %
            [\textsc{note}:~#2]}
        }
        
        % Color selected text (optional) and add a note from someone.
        % \notefrom[selected text]{from}{comments}
        % \notefrom{from}{comments}
        \newcommand{\notefrom}[3][]{
            {\color{green!50!black}#1 %
            [\textsc{from #2}:~#3]}
        }
        
        % Color selected text (optional) and add a note to someone.
        % \noteto[selected text]{to}{comments}
        % \noteto{to}{comments}
        \newcommand{\noteto}[3][]{
            {\color{red}#1 %
            [\textsc{to #2}:~#3]}
        }

        % Color and Line Settings for Boxed Text
        \mdfsetup{
        % middlelinecolor=red,
        middlelinewidth=1pt,
        % linecolor=blue,
        % linewidth=1pt,
        backgroundcolor=orange!10!white,
        linecolor=orange!50!black,
        roundcorner=5pt}
        
        % Shortcut for referencing figures/tables:
        % Usage:  \figref{fig:name} --> Figure 1.
        \newcommand{\figref}[1]{\figurename~\ref{#1}}
        \newcommand{\tabref}[1]{\tablename~\ref{#1}}
    % ++++++++++++++++++++++++++++++++++++++++++++++++++++++++++++++++++++++++

    % SETUP TikZ +++++++++++++++++++++++++++++++++++++++++++++++++++++++++++++
        \usetikzlibrary{arrows,shapes,matrix}
        \usetikzlibrary{decorations.pathmorphing} 
        \usepgflibrary{plotmarks}
        \usetikzlibrary{patterns}  
        \usetikzlibrary{positioning} 
        \usetikzlibrary{snakes}  
        \tikzstyle{block}=[draw opacity=0.7,line width=1.4cm]
        
        % MORE STUFF TO ADD HERE?
    % ++++++++++++++++++++++++++++++++++++++++++++++++++++++++++++++++++++++++

    % SETUP BIBLIOGRAPHY +++++++++++++++++++++++++++++++++++++++++++++++++++++
    % [This section MUST be used if you have a bibliography.    ]
    % [Otherwise, leave this section commented out.        ]
    % \begin{comment}

        % FIXME -- EXPLAIN
        
        % Setup the Bibliography Style -- Select ONE of the following:
        % \usepackage{natbib}
        % \usepackage[sectionbib,square]{natbib}     %%% See natbib.pdf for explanation.
        % \usepackage[sectionbib,round]{natbib}
        \usepackage[square,numbers]{natbib}

        \bibliographystyle{plainnat}

        % Natbib setup for author-year style
        % \bibpunct has 1 optional and 6 mandatory arguments:
        %  [0.] The character preceding a post-note, default is a comma plus space. In redefining this character, 
        %     one must include a space if one is wanted. 
        %  1. the opening bracket symbol, default = (
        %  2. the closing bracket symbol, default = )
        %  3. the punctuation between multiple citations, default = ;
        %  4. the letter `n' for numerical style, or `s' for numerical superscript style, 
        %    any other letter for author-year, default = author-year;
        %  5. the punctuation that comes between the author names and the year
        %  6. the punctuation that comes between years or numbers when common author lists are suppressed (default = ,);

        % Natbib setup for author-year style
        \bibpunct[, ]{(}{)}{,}{a}{}{,}                % Use author names
        % \bibpunct[, ]{[}{]}{,}{n}{}{,}            % Use numbers
        
        \def\bibfont{\small}
        \def\bibsep{\smallskipamount}
        \def\bibhang{24pt}
        \def\newblock{\ }
        \def\BIBand{and}
    % \end{comment}
    % ++++++++++++++++++++++++++++++++++++++++++++++++++++++++++++++++++++++++

    % DOCUMENT INFO ++++++++++++++++++++++++++++++++++++++++++++++++++++++++++
        \newcommand{\docTitle}{}

        % List authors here, separated by \and 
        \newcommand{\docAuthor}{}
        % \newcommand{\docAuthor}{}

        \newcommand{\docAffil}{
            Department of Industrial \& Systems Engineering,\\%
            University at Buffalo, Buffalo, New York, USA%
        }

        \newcommand{\docAbstract}{}

        \newcommand{\docKeyword}{}

        % This date will appear under the title.
        \newcommand{\docDate}{\today}       % {} --> don't show a date.
            
        % This date will appear in the page header:
        \newcommand{\draftDate}{\today}    % {\today} --> draft, {} --> finalized (hidden)
    
        % The image files should be saved here:
        \graphicspath{ {../../image/} }
    % ++++++++++++++++++++++++++++++++++++++++++++++++++++++++++++++++++++++++

    % DEFINE HEADER ++++++++++++++++++++++++++++++++++++++++++++++++++++++++++
        \usepackage{fancyhdr}
        \pagestyle{fancy}
        \ifthenelse{\equal{\draftDate}{}}
            {
                % This is the final version...remove the date from the header
                \chead{}
            }
            {
                % This is a working draft...include the date in the header
                % \chead{\color{red}DRAFT -- Updated \draftDate~at~\currenttime}
            }
        \lhead{}    % no left/right header content
        \rhead{}
        %\cfoot{}
        %\lfoot{}
        %\rfoot{}
        \renewcommand{\headrulewidth}{0pt}
        \renewcommand{\footrulewidth}{0pt}
        %\fancyfoot{}
    % ++++++++++++++++++++++++++++++++++++++++++++++++++++++++++++++++++++++++
    
    % DEFINE PROGRAMMING FORMAT ++++++++++++++++++++++++++++++++++++++++++++++
    \lstset{language=Python}          % Set your language (you can change the language for each code-block optionally)

    \definecolor{mygreen}{rgb}{0,0.6,0}
    \definecolor{mygray}{rgb}{0.5,0.5,0.5}
    \definecolor{mymauve}{rgb}{0.58,0,0.82}

    \lstset{ %
      backgroundcolor=\color{gray!05!white},   % choose the background color; you must add \usepackage{color} or \usepackage{xcolor}; should come as last argument
      basicstyle=\ttfamily,        % the size of the fonts that are used for the code
      breakatwhitespace=false,         % sets if automatic breaks should only happen at whitespace
      breaklines=true,                 % sets automatic line breaking
      captionpos=t,                    % sets the caption-position to bottom
      commentstyle=\color{black},    % comment style
      deletekeywords={...},            % if you want to delete keywords from the given language
      escapeinside={\%*}{*)},          % if you want to add LaTeX within your code
      extendedchars=true,              % lets you use non-ASCII characters; for 8-bits encodings only, does not work with UTF-8
      frame=single,                       % adds a frame around the code
      keepspaces=true,                 % keeps spaces in text, useful for keeping indentation of code (possibly needs columns=flexible)
      % keywordstyle=\color{blue},       % keyword style
      language=Python,                 % the language of the code
      morekeywords={*,...},           % if you want to add more keywords to the set
      numbers=none,                    % where to put the line-numbers; possible values are (none, left, right)
      numbersep=5pt,                   % how far the line-numbers are from the code
      % numberstyle=\tiny\color{mygray}, % the style that is used for the line-numbers
      rulecolor=\color{black},         % if not set, the frame-color may be changed on line-breaks within not-black text (e.g. comments (green here))
      showspaces=false,                % show spaces everywhere adding particular underscores; it overrides 'showstringspaces'
      showstringspaces=false,          % underline spaces within strings only
      showtabs=false,                  % show tabs within strings adding particular underscores
      stepnumber=1,                    % the step between two line-numbers. If it's 1, each line will be numbered
      % stringstyle=\color{mymauve},     % string literal style
      tabsize=4,                       % sets default tabsize to 2 spaces
      % title=\lstname,                   % show the filename of files included with \lstinputlisting; also try caption instead of title
      xleftmargin=35pt,
      xrightmargin=15pt, 
      aboveskip=0pt,
      belowskip=5pt
    }
    % ++++++++++++++++++++++++++++++++++++++++++++++++++++++++++++++++++++++++

    \newcommand{\titleSec}{
        % See https://tex.stackexchange.com/questions/216098/redefine-maketitle
        \begin{center}
        % \let \footnote \thanks
        {\Large \textbf{\docTitle} \par}

        % Authors?
        % Comment these lines out if you want to hide authors
        \vskip 1.0em%
        \lineskip .5em%
        \begin{tabular}[t]{c}
            \docAuthor
        \end{tabular}\par%

        % Affiliation?
        % Comment these lines out if you want to hide affiliation info
        \vskip 1.0em%
        {\small \docAffil \par}

        % Displayed date?
        % Comment these lines out if you want to hide the date
        %\vskip 1.0em%
        %{\small \docDate \par}  

        \end{center}
        \par
        \vskip 1.5em

        % \begin{abstract}
        %     \docAbstract
        % \end{abstract}

        % \begin{keywords}
        %     \docKeyword
        % \end{keywords}

        % This is version \texttt{\templateVersion} of this template.
        % Visit \templatesURL for the latest versions.
    }
\usepackage{makecell}
\usepackage{float}

\usetikzlibrary{shapes.geometric, arrows}
    \tikzstyle{startstop} = [rectangle, rounded corners, minimum width=3cm, minimum height=1cm,text centered, draw=black]
    \tikzstyle{io} = [trapezium, trapezium left angle=70, trapezium right angle=110, minimum width=3cm, minimum height=1cm, text centered, draw=black]
    \tikzstyle{process} = [rectangle, minimum width=2cm, minimum height=1cm, text centered, draw=black, inner sep=0.1cm]
    \tikzstyle{decision} = [diamond, minimum width=2cm, minimum height=0cm, text centered, draw=black, inner sep=0cm]
    \tikzstyle{arrow} = [thick,->,>=stealth]
    \tikzstyle{branchnode} = [circle, minimum size = 1cm, text centered, draw=black, inner sep=0.1cm]

\renewcommand{\docTitle}{Lecture 2 - Computational Complexity and Algorithm Design}
\renewcommand{\docAuthor}{Lan Peng, Ph.D.}
\renewcommand{\docAffil}{School of Management, Shanghai University, Shanghai, China}
\begin{document}
    \titleSec

    \begin{center}
        \texttt{``import numpy as np''}
    \end{center}

    \section{Preliminaries}
        \subsection{How difficult is to solve an integer program?}
            \begin{itemize}
                \item Hard!
                \item What is ``hard''? $\Rightarrow$ measurement of difficulties.
                \item $\Rightarrow$ computational complexity $\Rightarrow$ the art of classifying computational problems according to their resource usage.
                \item What is ``algorithm''? $\Rightarrow$ Given a string $s$, an algorithm takes the input $s$ and within finite number of steps, returns an output string $s^\prime$.
                \item What is ``resource usage''? $\Rightarrow$ How much \textit{time}/\textit{space} it takes for an algorithm to process $s$?
            \end{itemize}

        \subsection{Asymptotic Notation}
            \paragraph{$O$-Notation}
                Let $f(n)$ and $g(n)$ be real functions. We say $f(n) = O(g(n))$ if there exist positive constants $n_0$ and $c$ such that
                \begin{equation*}
                    0 \le f(n) \le cg(n), \quad \forall n \ge n_0
                \end{equation*}

                A common trick that finds $O(g(n))$ for $f(n)$ is to find function $g(n)$ such that 
                \begin{equation*}
                    \lim_{n \rightarrow \infty} \frac{f(n)}{g(n)} = C \quad \text{(constant)}
                \end{equation*}

                The $O$-Notation is referred as the asymptotic upper bound. The $O$-Notation is more widely used, since it can provide a clear estimate of the worst case time/space complexity of an algorithm that exclusively considers the critical grow factors.

            \paragraph{$\Omega$-Notation}
                Let $f(n)$ and $g(n)$ be real functions. We say $f(n) = \Omega(g(n))$ if there exist positive constants $n_0$ and $c$ such that
                \begin{equation*}
                    0 \le cg(n) \le f(n), \quad \forall n \ge n_0
                \end{equation*}

                The $\Omega$-Notation is referred as the asymptotic lower bound. In addition:

                \begin{theorem}
                    $f(n) = O(g(n)) \Leftrightarrow g(n) = \Theta(f(n))$
                \end{theorem}
            
            \paragraph{$\Theta$-Notation}
                Let $f(n)$ and $g(n)$ be real functions. We say $f(n) = \Theta(g(n))$ if there exist positive constants $n_0$, $c_1$, and $c_2$ such that
                \begin{equation*}
                    c_1g(n) \le f(n) \le c_2g(n), \quad \forall n \ge n_0
                \end{equation*}

                The $\Theta$-Notation is referred as the asymptotic tight bound. In addition:

                \begin{theorem}
                    $f(n) = \Theta(g(n)) \Leftrightarrow f(n) = O(g(n)) \text{ and } f(n) = \Omega(g(n))$
                \end{theorem}

    \section{P v.s. NP}
        \subsection{P and NP}
            \paragraph{Decision problem} A decision problem $X$ is the set of strings on which the output is yes, i.e., $s \in X$ iff the correct output for the input $s$ is 1 (yes).

                \begin{example}
                    The optimization version of the knapsack problem: Given a set of items $N$, each item $i \in N$ having a weight $w_i$ and a value $c_i$, a capacity $b$ and a threshold value $k$, find a collection of items $S \subseteq N$ of the maximum value whose total weight is less than or equal to $b$. The output is a set $S$.

                    The decision version of the knapsack problem: Given a set of items $N$, each item $i \in N$ having a weight $w_i$ and a value $c_i$, a capacity $b$ and a threshold value $k$, determine if there exists a collection of items $S \subseteq N$ whose total weight is less than or equal to $b$ and its total value is at least $k$. The output is a boolean value, True/False.
                \end{example}

                If we have an algorithm to solve the optimization version of knapsack problem, we can immediately solve the decision version. 

            \paragraph{Class $P$}
                \begin{definition}[polynomial running time]
                    Algorithm $A$ has polynomial running time if there is a polynomial function $p(\dot)$ such that for every string $s$, $A$ terminates on $s$ in at most $p(|s|)$ steps.
                \end{definition}

                \begin{definition}[Class $P$]
                    The complexity class $P$ is the set of decision problems $X$ that can be solved in polynomial time.
                \end{definition}

                That is, there is a known algorithm that provides the solution of any instance of size $n$ in time $n^{O(1)}$.

                \begin{example} The following problems are known as in class $P$:
                    \begin{itemize}
                        \item Shortest path problem
                        \item Maximum flow problem
                        \item Spanning tree problem
                        \item Linear programming (but not the simplex method)
                        \item Assignment problem
                    \end{itemize}
                \end{example}

            \paragraph{Class NP}
                \begin{definition}[certifier, certificate]
                    $B$ is an efficient certifier for a problem $X$ if
                    \begin{itemize}
                        \item $B$ is a polynomial-time algorithm that takes two input strings $s$ and $t$
                        \item There is a polynomial function $p$ such that $s \in X$ if and only if there is a string $t$ such that $|t| \le p(|s|)$ and $B(s, t) = 1$
                    \end{itemize}
                    The string $t$ such that $B(s, t) = 1$ is called a certificate.
                \end{definition}

                \begin{example}
                    Q: Given a graph $G = (V, E)$ with a Hamiltonian cycle, how can one convince another there exists a Hamiltonian cycle? 

                    A: One can send a string (certificate) to another, the other will run an algorithm (certifier) to check if the string represents a Hamiltonian cycle.
                \end{example}

                \begin{definition}[class NP]
                    The complexity class NP is the set of all problems for which there exists an polynomial-time certifier.
                \end{definition}

            \paragraph{P v.s. NP}
                \begin{itemize}
                    \item Is $P = NP?$ This is the most famous and fundamental open problem in computer science.
                    \item Most people believe $P \neq NP$.
                    \item If $P = NP$, that means if one can check a solution in polynomial time, one can solve it in polynomial time.
                    \item We are sure that $P \subseteq NP$
                \end{itemize}

        \subsection{Polynomial Time Reduction}
            \paragraph{Polynomial-time reducible} Given a black box algorithm $A$ that solves a problem $X$, if any instance of a problem $Y$ can be solved using a polynomial number of standard computational steps, plus a polynomial number of calls to $A$, then we say $Y$ is polynomial-time reducible to $X$, denoted as $Y \le_P X$.

            Suppose $Y \le_P X$:
            \begin{itemize}
                \item If $X$ can be solved in polynomial time, then $Y$ can be solved in polynomial time.
                \item If $Y$ cannot be solved in polynomial time, then $X$ cannot be solved in polynomial time.
            \end{itemize}

            $X$ is at least as hard as $Y$.

        \subsection{NP-Completeness}
            \paragraph{NP-Completeness}
                A problem $X$ is called NP-complete if
                \begin{itemize}
                    \item $X \in NP$, and
                    \item $Y \le_P X$, for every $Y \in NP$
                \end{itemize}

                If any NP-complete problem can be solved in polynomial time, then $P = NP$. Unless $P = NP$, a NP-complete problem cannot be solved in polynomial time.

            \paragraph{Circuit-Sat}
                The Circuit Satisfiability problem is the first NP-complete problem.

                \begin{theorem}[Cook's Theorem]
                    SAT is NP-Complete
                \end{theorem}

                \begin{figure}[H]
                    \centering
                    \includegraphics[width=0.5\textwidth]{../../image/circuit.png}
                \end{figure}

                \begin{itemize}
                    \item key fact: any algorithm that takes $n$ bits as input and outputs $0/1$ with running time $T(n)$ can be converted into a circuit of size $p(T(n))$ for some polynomial function $p(\cdot)$.
                    \item Then, we can show that any problem $Y \in NP$ can be reduced to Circuit-Sat
                \end{itemize}

            \paragraph{Reductions of NP-Complete problems}
                We will show a SAT $\le_P$ 3-SAT

                \begin{definition}[3-CNF]
                    3-CNF is a special case of formula:
                    \begin{itemize}
                        \item Boolean variables: $x_1, x_2, \cdots, x_n$
                        \item Literals: $x_i$ or $\bar{x_i}$
                        \item Clause: disjunction (``or'') of at most 3 literals
                        \item 3-CNF formula: conjunctino (``and'') of clauses.
                    \end{itemize}
                \end{definition}

                \begin{example}
                    This is a 3-CNF: $(x_1 \vee x_2 \vee x_3) \wedge (x_2 \vee \bar{x_3} \vee x_4)$
                \end{example}

                To satisfy a 3-CNF, we need to satisfy all clauses. To satisfy a clause, we need to satisfy at least 1 literal. Associate every wire with a new variable, the circuit will be equivalent to a formula. Each formula can be transformed into a 3-CNF. 

                The SAT is satisfiable iff the 3-CNF is satifiable, and the size of the 3-CNF formula is polynomial in the size of the circuit. Thus SAT $\le_P$ 3-SAT.

                For other problems, here is a polynomial-reducible relation graph for reference.

                \begin{figure}[H]
                    \centering
                    \includegraphics[width=0.8\textwidth]{../../image/polyRedu.png}
                \end{figure}

    \section{Algorithm Analysis and Design}
        \subsection{Three programming paradigms}
            \paragraph{Greedy Algorithm}
                \begin{itemize}
                    \item Make a greedy choice
                    \item At each step, make an irrevocable decision using a ``reasonable'' strategy
                    \item Prove that the greedy choice is safe
                    \item Show that the remaining task after applying the strategy is to solve a/many \textbf{smaller instance}(s) of the same problem
                    \item Usually for optimization problems.
                \end{itemize}

            \paragraph{Dynamic Programming}
                \begin{itemize}
                    \item Break up a problem into many \textbf{overlapping} sub-problems
                    \item Build solutions for larger and larger sub-problems
                    \item Use a table/dictionary to store solutions for sub-problems for reuse
                \end{itemize}

            \paragraph{Divide and Concur}
                \begin{itemize}
                    \item Break a problem into many \textbf{independent} sub-problems
                    \item Solve each sub-problem separately
                    \item Combine solutions for sub-problems to form a solution for the original one
                    \item Usually used to design more efficient algorithms
                \end{itemize}

        \subsection{Master Theorem}
            \begin{theorem}
                $T(n) = aT(n/b) + O(n^c)$, where $a \ge 1, b > 1, c \ge 0$ are constants. Then,
                \begin{equation*}
                    T(n) = \begin{cases}
                        & O(n^{\log_b a}) \quad \text{if } c < \log_b a\\
                        & O(n^c \log n) \quad \text{if } c = \log_b a\\
                        & O(n^c) \quad \text{if } c > \log_b a
                    \end{cases}
                \end{equation*}
            \end{theorem}

            \begin{figure}[H]
                \centering
                \includegraphics[width=0.8\textwidth]{../../image/masterTheorem.png}
                \label{fig:masterTheorem}
            \end{figure}

            \begin{itemize}
                \item $c < \log_b a$: bottom-level dominates, $(\frac{a}{b^c})^{\log_b n} n^c = O(n^{\log_b a})$
                \item $c = \log_b a$: all levels have same time, $n^c \log_b n = O(n^c \log n)$
                \item $c > \log_b a$: top-level dominates, $O(n^c)$
            \end{itemize}

    \section{Some examples}
        \subsection{Minimum Spanning Tree Problem}
            \paragraph{Basic Concepts}
                \begin{example}
                    A company wants to build a communication network for their offices. For a link between office $v$ and office $w$, there is a cost $c_{vw}$. If an office is connected to another office, then they are connected to with all its neighbors. Company wants to minimize the cost of communication networks.
                \end{example}

                \begin{definition}[Minimum spanning tree problem]
                    Given a connected graph graph $G$, and a cost $C_e, \forall e\in E$, find a minimum cost spanning tree of $G$
                \end{definition}

            \paragraph{Kruskal's Algorithm}
                Let's start with a greedy algorithm

                \begin{algorithm}[H]
                    \caption{MST-Greedy(G, w)}
                    \begin{algorithmic}[1]
                        \State $F = \emptyset$
                        \State Sort edges in $E$ in non-decreasing order of weight $w$
                        \For {$e = (u, v)$}
                            \If {$u$ and $v$ are not connected by a path of edge in $F$}
                                \State $F = F \cup \{(u, v)\}$
                            \EndIf
                        \EndFor
                        \State \Return $F$
                    \end{algorithmic}
                \end{algorithm}


                The Kruskal's algorithm is as follows
                \begin{algorithm}[H]
                    \caption{Kruskal's Algorithm}
                    \begin{algorithmic}
                        \State $F \gets \emptyset$
                        \State $S \gets \{\{v\}: v \in V\}$
                        \State Sort edges in $E$ in non-decreasing order of weight $w$
                        \For {$e = (u, v)$}
                            \State $S_u \gets$ the set in $S$ containing $u$
                            \State $S_v \gets$ the set in $S$ containing $v$
                            \If {$S_u \neq S_v$}
                                \State $F \gets F \cup \{(u, v)\}$
                                \State $S \gets S \setminus \{S_u\} \setminus \{S_v\} \cup \{S_u \cup S_v\}$
                            \EndIf
                        \EndFor
                        \State \Return $F$
                    \end{algorithmic}
                \end{algorithm}

            \paragraph{Prim's Algorithm}
                Let's start with another greedy algorithm
                \begin{algorithm}[H]
                    \caption{MST-Greedy(G, w)}
                    \begin{algorithmic}
                        \State $S \gets \{s\}$, where $s$ is arbitary vertex in $V$
                        \State $F \gets \emptyset$
                        \While {$S \neq V$}
                            \State $(u, v) \gets$ the lightest edge between $S$ and $V \setminus S$, where $u \in S$ and $v \in V \setminus S$
                            \State $S \gets S \cup \{v\}$
                            \State $F \gets F \cup \{(u, v)\}$
                        \EndWhile
                    \end{algorithmic}
                \end{algorithm}

                The Prim's algorithm is as follows
                \begin{algorithm}[H]
                    \caption{Prim's Algorithm}
                    \begin{algorithmic}
                        \State $s \gets$ arbitary vertex in $G$
                        \State $S \gets \emptyset$, $d(s) \gets 0$ and $d(v) \gets \infty$ for every $v \in V \setminus \{s\}$
                        \While {$S \neq V$}
                            \State $u \gets $ vertex in $V\setminus S$ with minimum $d(u)$
                            \State $S \gets S \cup \{u\}$
                            \For {Each $v \in V \setminus S$ such that $(u, v) \in E$}
                                \If {$w(u, v) < d(v)$}
                                    \State $d(v) \gets w(u, v)$
                                    \State $\pi(v) \gets u$
                                \EndIf
                            \EndFor
                        \EndWhile
                        \State \Return $\{(u, \pi(u))| u \in V \setminus \{s\}\}$
                    \end{algorithmic}
                \end{algorithm}

        \subsection{Knapsack Problem}
            \paragraph{Integer program model}
                \begin{align}
                    \max \quad & \sum_{i \in S} v_i\\
                    \text{s.t.} \quad & \sum_{i \in S} w_i \le W\\
                    & w_i \ge 0, \forall i \in S
                \end{align}

            \paragraph{Solve knapsack problem by dynamic programming}
                \begin{itemize}
                    \item Let $opt[i, W^\prime]$ be the optimum value when budget is $W^\prime$ and the items are $\{1, 2, 3, \cdots, i\}$.
                    \item If $i = 0, opt[i, W^\prime] = 0$ for every $W^\prime = 0, 1, \cdots, W$. Then
                    \item $opt[i, W'] = \begin{cases}
                        & 0, \qquad i = 0\\
                        & opt[i - 1, W^\prime], \qquad i > 0, w_i > W^\prime \\
                        & \max \left\{ opt[i - 1, W^\prime], opt[i - 1, W^\prime - w_i] + v_i \right\}, \qquad i > 0, w_i \le W^\prime
                    \end{cases}$
                \end{itemize}
                
        \subsection{Fibonacci Numbers}
            \paragraph{Fibonacci sequence}
                \begin{equation}
                    0, 1, 1, 2, 3, 5, 8, 13, 21, 34, 55, 89, \cdots
                \end{equation}

            \paragraph{Naive algorithm}
                Use recursion (in a stupid way)
                \begin{algorithm}[H]
                    \caption{Fib(n)}
                    \begin{algorithmic}[1]
                        \If {$n = 0$}
                            \State \Return 0
                        \EndIf
                        \If {$n = 1$}
                            \State \Return 1
                        \EndIf
                        \State \Return Fib($n - 1$) + Fib($n - 2$)
                    \end{algorithmic}
                \end{algorithm}

                The runtime is exponential $O(2^n)$. This is stupid.

            \paragraph{Reasonable algorithm}
                Use dynamic programming
                \begin{algorithm}[H]
                    \caption{Fib(n)}
                    \begin{algorithmic}[1]
                        \State F(0) $\gets$ 0
                        \State F(1) $\gets$ 1
                        \For {$i \gets 2$ to $n$}
                            \State F($i$) $\gets$ F($i - 1$) + F($i - 2$)
                        \EndFor
                        \State \Return F(n)
                    \end{algorithmic}
                \end{algorithm}

                The runtime is $O(n)$.

            \paragraph{Even better algorithm}
                
                Notice that

                \begin{align*}
                    \begin{bmatrix}
                        F_n \\ F_{n - 1}
                    \end{bmatrix} &= \begin{bmatrix}
                        1 & 1 \\ 1 & 0
                    \end{bmatrix} \begin{bmatrix}
                        F_{n - 1} \\ F_{n - 2}
                    \end{bmatrix}\\
                    \begin{bmatrix}
                        F_n \\ F_{n - 1}
                    \end{bmatrix} &= \begin{bmatrix}
                        1 & 1 \\ 1 & 0
                    \end{bmatrix}^2 \begin{bmatrix}
                        F_{n - 2} \\ F_{n - 3}
                    \end{bmatrix}\\
                    &\cdots\\
                    \begin{bmatrix}
                        F_n \\ F_{n - 1}
                    \end{bmatrix} &= \begin{bmatrix}
                        1 & 1 \\ 1 & 0
                    \end{bmatrix}^{n - 1} \begin{bmatrix}
                        F_1 \\ F_0
                    \end{bmatrix}\\
                \end{align*}

                First define power($n$)

                \begin{algorithm}[H]
                    \caption{power(n)}
                    \begin{algorithmic}[1]
                        \If {$n = 0$}
                            \State \Return $\begin{bmatrix}
                                1 & 0 \\ 0 & 1
                        \end{bmatrix}$
                        \EndIf
                        \State $R \gets$ power($n/2$.floor())
                        \State $R \gets R \times R$
                        \If {$n$ is odd number}
                            \State $R \gets R \times \begin{bmatrix}
                                1 & 1 \\ 1 & 0
                            \end{bmatrix}$
                        \EndIf
                        \State \Return $R$
                    \end{algorithmic}
                \end{algorithm}

                Then the Fibonacci sequence is calculated by

                \begin{algorithm}[H]
                    \caption{Fib(n)}
                    \begin{algorithmic}[1]
                        \If {$n = 0$}
                            \State \Return 0
                        \EndIf
                        \State $M \gets $ power($n - 1$)
                        \State \Return M[1][1]
                    \end{algorithmic}
                \end{algorithm}

                The time complexity is $T(n) = T(n/2) + O(1) = O(\log n)$

        \subsection{Multiplications}
            \paragraph{Polynomial Multiplication}
                Given two polynomials of degree $n - 1$, we need to find the product of two polynomials.

                \begin{example}
                    Let $A = (4, -5, 2, 3)$ and $B = (-5, 6, -3, 2)$ as input. The expected output $C = (-20, 49, -52, 20, 2, -5, 6)$, since
                    \begin{align*}
                        &(3 x^3 + 2 x^2 - 5 x + 4) \times (2 x^3 - 3 x^2 + 6x - 5)\\
                       =& 6 x^6 - 9 x^5 + 18 x^4 - 15 x^3 + 4 x^5 - 6x^4 + 12x^3 - 10 x^2\\
                        & -10 x^4 + 15 x^3 - 30 x^2 + 25 x + 8 x^3 - 12 x^2 + 24 x - 20\\
                       =& 6 x^6 - 5 x^5 + 2x^4 + 20x^3 - 52 x^2 + 49 x - 20
                    \end{align*}
                \end{example}

                A naive algorithm for polynomial multiplication is as follows
                \begin{algorithm}[H]
                    \caption{naivePolyMultiply($A, B, n$)}
                    \begin{algorithmic}[1]
                        \State Let $C[k] = 0$ for every $k = 0, 1, \cdots, 2n - 2$
                        \For {$i \gets 0$ to $n - 1$}
                            \For {$j \gets 0$ to $n - 1$}
                                \State $C[i + j] \gets C[i + j] + A[i] \times B[j]$
                            \EndFor
                        \EndFor
                        \State \Return $C$
                    \end{algorithmic}
                \end{algorithm}

                Obviously, the running time is $O(n^2)$. Now we try to use divide-and-conquer to improve. For a polynomial $p(x)$ with degree of $n - 1$, denote
                \begin{equation*}
                    p(x) = p_H(x) x^{\frac{n}{2}} + p_L(x)
                \end{equation*}

                then, $p_H(x)$ and $p_L(x)$ are polynomials of degree $n/2 - 1$. Consider
                \begin{equation*}
                    pq = (p_H x^{\frac{n}{2}} + p_L)(q_H x^{\frac{n}{2}} + q_L) = p_Hq_Hx^n + (p_Hq_L + p_Lq_H)x^{\frac{n}{2}} + p_Lq_L
                \end{equation*}

                The recurrence time is
                \begin{equation*}
                    T(n) = 4T(n/2) + O(n) = O(n^2)
                \end{equation*}

                still not good... Consider
                \begin{align*}
                    pq &= (p_H x^{\frac{n}{2}} + p_L)(q_H x^{\frac{n}{2}} + q_L) \\
                       &= p_Hq_Hx^n + ((p_H + p_L)(q_H + q_L) - p_Hq_H - p_Lq_L)x^{\frac{n}{2}} + p_Lq_L
                \end{align*}

                The new recurrence time is
                \begin{equation*}
                    T(n) = 3T(n/2) + O(n) = O(n^{\log_2 3}) = O(n^{1.585})
                \end{equation*}

                Better!!! The algorithm is given as follows

                \begin{algorithm}[H]
                    \caption{polyMultiply($A, B, n$)}
                    \label{algo:polyMultiply}
                    \begin{algorithmic}[1]
                        \State $A_L \gets A[0 .. n/2-1]$, $A_H \gets A[n/2 .. n-1]$
                        \State $B_L \gets B[0 .. n/2-1]$, $B_H \gets B[n/2 .. n-1]$
                        \State $C_L \gets polyMultiply(A_L, B_L, n/2)$
                        \State $C_H \gets polyMultiply(A_H, B_H, n/2)$
                        \State $C_M \gets polyMultiply(A_L + A_H, B_L + B_H, n/2)$
                        \State Initialize $C \gets [0 .. 0]$ ($2n - 1$) 0s
                        \For {$i \gets 0$ to $n - 2$}
                            \State $C[i] \gets C[i] + C_L[i]$
                            \State $C[i + n] \gets C[i + n] + C_H[i]$
                            \State $C[i + n/2] \gets C[i + n/2] + C_M[i] - C_L[i] - C_H[i]$
                        \EndFor
                        \State \Return $C$
                    \end{algorithmic}
                \end{algorithm}

                Notice that the name of Algorithm \ref{algo:polyMultiply} appears inside itself (line 3, 4, 5). This is referred as \textit{recursion}. The key to recursion functions is to break complex problems into smaller instances of the \textbf{same} problems, so we need to ``think beyond timescape''.

            \paragraph{Matrix multiplication}
                Given two matrices $\mathbf{A} \in \mathbb{R}^{n\times}$ and $\mathbf{B} \in \mathbb{R}^{n\times}$, we need to find the production, $\mathbf{C} = \mathbf{AB} \in \mathbb{R}^{n\times}$.

                A naive algorithm for matrix multiplication is as follows

                \begin{algorithm}[H]
                    \caption{naiveMatMultiply(A, B, n)}
                    \begin{algorithmic}[1]
                        \For {$i \gets 1$ to $n$}
                            \For {$j \gets 1$ to $n$}
                                \State $C[i, j] \gets 0$
                                \For {$k \gets 1$ to n}
                                    \State $C[i, j] \gets C[i, j] + A[i, k] \times B[k, j]$
                                \EndFor
                            \EndFor
                        \EndFor
                        \State \Return $C$
                    \end{algorithmic}
                \end{algorithm}

                The running time is $O(n^3)$. Try divided-and-conquer. Let

                \begin{align*}
                    \mathbf{A} = \begin{bmatrix}
                        \mathbf{A}_{11} & \mathbf{A}_{12}\\
                        \mathbf{A}_{21} & \mathbf{A}_{22}
                    \end{bmatrix}, \quad
                    \mathbf{B} = \begin{bmatrix}
                        \mathbf{B}_{11} & \mathbf{B}_{12}\\
                        \mathbf{B}_{21} & \mathbf{B}_{22}
                    \end{bmatrix}
                \end{align*}

                each sub-matrix is $n/2 \times n/2$. Then

                \begin{align*}
                    \mathbf{C} = \begin{bmatrix}
                        \mathbf{A}_{11}\mathbf{B}_{11} + \mathbf{A}_{12}\mathbf{B}_{21} &
                        \mathbf{A}_{11}\mathbf{B}_{12} + \mathbf{A}_{12}\mathbf{B}_{22} \\
                        \mathbf{A}_{21}\mathbf{B}_{11} + \mathbf{A}_{22}\mathbf{B}_{21} & 
                        \mathbf{A}_{21}\mathbf{B}_{12} + \mathbf{A}_{22}\mathbf{B}_{22}
                    \end{bmatrix}
                \end{align*}

                Still not good enough because $T(n) = 8T(n/2) + O(n)$. 

            \paragraph{Strassen's Algorithm}

                In 1969, the first matrix multiplication algorithm with time complexity lower than $O(n^3)$ was introduced by Volker Strassen, the famous Strassen's Algorithm. 

                For $\mathbf{A}, \mathbf{B}, \mathbf{C}$, let
                \begin{align*}
                    \mathbf{M}_1 &= (\mathbf{A}_{11} + \mathbf{A}_{22}) (\mathbf{B}_{11} + \mathbf{B}_{22})\\
                    \mathbf{M}_2 &= (\mathbf{A}_{21} + \mathbf{A}_{22}) \mathbf{B}_{11}\\
                    \mathbf{M}_3 &= \mathbf{A}_{11} (\mathbf{B}_{12} - \mathbf{B}_{22})\\
                    \mathbf{M}_4 &= \mathbf{A}_{22} (\mathbf{B}_{21} - \mathbf{B}_{11})\\
                    \mathbf{M}_5 &= (\mathbf{A}_{11} + \mathbf{A}_{12}) \mathbf{B}_{22}\\
                    \mathbf{M}_6 &= (\mathbf{A}_{21} - \mathbf{A}_{11}) (\mathbf{B}_{11} + \mathbf{B}_{12}) \\
                    \mathbf{M}_7 &= (\mathbf{A}_{12} - \mathbf{A}_{22}) (\mathbf{B}_{21} + \mathbf{B}_{22})
                \end{align*}

                Then, 

                \begin{align*}
                    \mathbf{C}_{11} &= \mathbf{M}_1 + \mathbf{M}_4 - \mathbf{M}_5 + \mathbf{M}_7\\
                    \mathbf{C}_{12} &= \mathbf{M}_3 + \mathbf{M}_5\\
                    \mathbf{C}_{21} &= \mathbf{M}_2 + \mathbf{M}_4\\
                    \mathbf{C}_{22} &= \mathbf{M}_1 - \mathbf{M}_2 + \mathbf{M}_3 + \mathbf{M}_6
                \end{align*}

                In total, $T(n) = 7T(n/2) + O(n) = O(n^{\log_2 7})$.

                Why is this important? Matrix multiplication has been widely used in machine learning such as calculating convolutions. For $n > 300$, the Strassen's algorithm will run significantly faster than the naive algorithm.

                There are some algorithms even faster than Strassen, e.g., the Coppersmith-Winograd matrix multiplication algorithm.
    % \bibliography{literature}
\end{document}