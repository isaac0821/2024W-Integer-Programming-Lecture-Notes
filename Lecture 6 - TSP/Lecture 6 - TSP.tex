\documentclass[12pt,oneside]{book}

	% History ================================================================
	% 2023.06.03 - Modified from Chase Murray's version
	% ========================================================================

    % STANDARD PACKAGES ======================================================
    \usepackage{datetime}
    \usepackage{graphicx}
    % \usepackage{ctex} % Allow Chinese characters
    \usepackage[utf8]{inputenc}
    \usepackage[american]{babel}
    \usepackage{amssymb}
    \usepackage[intlimits]{amsmath}
    \usepackage{amsfonts}
    \usepackage{amsthm}
    \usepackage{array}
    \usepackage{mdwlist}        
    % \usepackage[labelsep=quad,indention=10pt]{subfig}
    \usepackage{algorithm}
    \usepackage[noend]{algpseudocode}
    \usepackage{lscape}
    \usepackage{rotating} % Allows \begin{sideways} \end{sideways} for vertical table headers.    
    \usepackage{threeparttable} % Allow footnotes in tables.    
    \usepackage{tabularx}
    \usepackage{multirow} % Allow table cells to span multiple rows/cols.
    \usepackage{makecell}
    \usepackage{longtable}
    \usepackage{url} % Allow \url{} and \href{url}{name}
    \usepackage{verbatim}    
    \usepackage{enumerate} % http://www.tex.ac.uk/cgi-bin/texfaq2html?label=enumerate
    \usepackage{color} % Allow colored fonts
    \usepackage[toc,page]{appendix}

    \usepackage{bm}

    \usepackage{tikz}
        \usetikzlibrary{shapes.geometric, arrows}
            \tikzstyle{startstop} = [rectangle, rounded corners, minimum width=3cm, minimum height=1cm,text centered, draw=black]
            \tikzstyle{io} = [trapezium, trapezium left angle=70, trapezium right angle=110, minimum width=3cm, minimum height=1cm, text centered, draw=black]
            \tikzstyle{process} = [rectangle, minimum width=2cm, minimum height=1cm, text centered, draw=black, inner sep=0.1cm]
            \tikzstyle{decision} = [diamond, minimum width=2cm, minimum height=0cm, text centered, draw=black, inner sep=0cm]
            \tikzstyle{arrow} = [thick,->,>=stealth]
            \tikzstyle{branchnode} = [circle, minimum size = 1cm, text centered, draw=black, inner sep=0.1cm]
            \tikzstyle{solidNode} = [circle, minimum size = 0.1cm, fill=black]
            \tikzstyle{link} = [thick, -]
            \tikzstyle{matchedLink} = [decorate, decoration={snake}]
            \tikzstyle{circleNode} = [
                circle, 
                minimum size = 0.7cm, 
                text centered, 
                draw=black, 
                inner sep=0.1cm
            ]
    \usepackage{diagbox}
    \usepackage{lastpage} % \pageref{LastPage} = total number of pages.
    \usepackage{ifthen}        
    \usepackage{setspace} % Allows \singlespacing, \onehalfspacing, \doublespacing 
    \usepackage{listings} % Allows formatting of Python code (and other languages)
    % \usepackage{wrapfig}
    \usepackage[normalem]{ulem} % Allows strikethrough (\sout{text to strike})
    % \usepackage{subfigure}        % Allows subfigs/subfloats


    \usepackage{xcolor,colortbl}    % http://ctan.org/pkg/xcolor
    % \usepackage[table]{xcolor}    % https://tex.stackexchange.com/questions/50349/color-only-a-cell-of-a-table
    
    % Make sure that {color} and {xcolor} are called before mdframed
    \usepackage[framemethod=TikZ]{mdframed}    % Allows colored textbox

    \usepackage{lipsum}                     % Dummy text
    % ========================================================================

    % DEFINE PROGRAMMING FORMAT ++++++++++++++++++++++++++++++++++++++++++++++
        \lstset{language=Python}          % Set your language (you can change the language for each code-block optionally)

        \definecolor{mygreen}{rgb}{0,0.6,0}
        \definecolor{mygray}{rgb}{0.5,0.5,0.5}
        \definecolor{mymauve}{rgb}{0.58,0,0.82}

        \lstset{
          backgroundcolor=\color{gray!05!white},   % choose the background color; you must add \usepackage{color} or \usepackage{xcolor}; should come as last argument
          basicstyle=\ttfamily,                    % the size of the fonts that are used for the code
          breakatwhitespace=false,                 % sets if automatic breaks should only happen at whitespace
          breaklines=true,                         % sets automatic line breaking
          captionpos=t,                            % sets the caption-position to bottom
          commentstyle=\color{black},              % comment style
          deletekeywords={...},                    % if you want to delete keywords from the given language
          escapeinside={\%*}{*)},                  % if you want to add LaTeX within your code
          extendedchars=true,                      % lets you use non-ASCII characters; for 8-bits encodings only, does not work with UTF-8
          frame=single,                               % adds a frame around the code
          keepspaces=true,                         % keeps spaces in text, useful for keeping indentation of code (possibly needs columns=flexible)
          % keywordstyle=\color{blue},             % keyword style
          language=Python,                         % the language of the code
          morekeywords={*,...},                    % if you want to add more keywords to the set
          numbers=left,                            % where to put the line-numbers; possible values are (none, left, right)
          numbersep=5pt,                           % how far the line-numbers are from the code
          % numberstyle=\tiny\color{mygray},       % the style that is used for the line-numbers
          rulecolor=\color{black},                 % if not set, the frame-color may be changed on line-breaks within not-black text (e.g. comments (green here))
          showspaces=false,                        % show spaces everywhere adding particular underscores; it overrides 'showstringspaces'
          showstringspaces=false,                  % underline spaces within strings only
          showtabs=false,                          % show tabs within strings adding particular underscores
          stepnumber=1,                            % the step between two line-numbers. If it's 1, each line will be numbered
          % stringstyle=\color{mymauve},           % string literal style
          tabsize=4,                               % sets default tabsize to 2 spaces
          % title=\lstname,                        % show the filename of files included with \lstinputlisting; also try caption instead of title
          xleftmargin=35pt,
          xrightmargin=15pt, 
          aboveskip=0pt,
          belowskip=5pt
        }
    % ++++++++++++++++++++++++++++++++++++++++++++++++++++++++++++++++++++++++

    % DEFINE/RENEW SOME ENVIRONMENTS =========================================    
        % \renewenvironment{abstract}
        %   {\normalfont\footnotesize
        %     \list{}{\labelwidth0pt
        %       \leftmargin20pt \rightmargin\leftmargin
        %       \listparindent\parindent \itemindent0pt
        %       \parsep0pt
        %       \let\fullwidthdisplay\relax
        %     }
        %     \item[\hskip\labelsep\bfseries\abstractname:] %
        % }{
        %   \endlist}

        % \newcommand{\keywordsname}{Keywords}
        % \newenvironment{keywords}
        %   {\normalfont\footnotesize
        %     \list{}{\labelwidth0pt
        %       \leftmargin20pt \rightmargin\leftmargin
        %       \listparindent\parindent \itemindent0pt
        %       \parsep0pt
        %       \let\fullwidthdisplay\relax}
        %     \item[\hskip\labelsep\bfseries\keywordsname:]}{\endlist}

        % \newcommand{\dochistname}{History}
        % \newenvironment{DocHistory}
        %   {\normalfont\footnotesize
        %     \list{}{\labelwidth0pt
        %       \leftmargin20pt \rightmargin\leftmargin
        %       \listparindent\parindent \itemindent0pt
        %       \parsep0pt
        %       \let\fullwidthdisplay\relax}
        %     \item[\hskip\labelsep\bfseries\dochistname:]}{\endlist}
    % ========================================================================    

    % DEFINE PAGE FORMATTING +++++++++++++++++++++++++++++++++++++++++++++++++
        % Select Line Spacing:
        \singlespacing
        % \onehalfspacing        
        % \doublespacing    

        % Margins:
        \usepackage[letterpaper,left=1.0in,top=1.0in,right=1.0in,bottom=1.0in]{geometry}
    
        % Page Style
        \pagestyle{plain}    % Includes page number
        %\pagestyle{empty}    % Completely blank                

        % By default all math is set to inline mode. The \displaystyle command
        % ensures that we don't get small fractions or summations with limits
        % on the sides.
        \everymath{\displaystyle}    
        
        % http://tex.stackexchange.com/questions/5223/command-for-argmin-or-argmax
        \DeclareMathOperator*{\argmin}{arg\,min}

        % Allow flalign items to be split over multiple pages:
        \allowdisplaybreaks[1]   % See ftp://ftp.ams.org/pub/tex/doc/amsmath/amsldoc.pdf    
    % ++++++++++++++++++++++++++++++++++++++++++++++++++++++++++++++++++++++++

    % DEFINITION, THEOREM, AND LEMMA +++++++++++++++++++++++++++++++++++++++++

        \theoremstyle{definition}
            \newtheorem{definition}{Definition}[section]
            \newtheorem*{example}{Example}
            \newtheorem{problem}{Problem}[section]
            \newtheorem*{solution}{Solution}
            \newtheorem{hypothesis}{Hypothesis}[section]
        \theoremstyle{plain}
            \newtheorem{theorem}{Theorem}[section]
            \newtheorem{corollary}{Corollary}[theorem]
            \newtheorem{lemma}[theorem]{Lemma}
            \newtheorem{conjecture}{Conjecture}
            \newtheorem{proposition}{Proposition}
        \theoremstyle{remark}
            \newtheorem*{remark}{Remark}

    % ++++++++++++++++++++++++++++++++++++++++++++++++++++++++++++++++++++++++

    % CUSTOM MACROS ++++++++++++++++++++++++++++++++++++++++++++++++++++++++++

        % This is how you may create a new variable:
        % \newcommand{\docjunk}{ text to display }
        
        % See https://gist.github.com/benkehoe/c46647134d4bbd514869
        % for more examples.

        % Create a box marked ``To Do'' around text.
        % \todo{  insert text here  }.
        \newcommand{\todo}[1]{\vspace{5 mm}\par \noindent
        \marginpar{\textsc{to do}}
        \framebox{\begin{minipage}[c]{0.95 \textwidth}
        \tt\begin{center} #1 \end{center}\end{minipage}}\vspace{5 mm}\par}

        % Create an empty box marked ``Result'' in the margin.
        % Specify the number of empty rows.
        % \result{8 em}.
        \newcommand{\result}[1]{\vspace{5 mm}\par \noindent
        \marginpar{\textsc{Result}} $\qquad\qquad$
        \framebox{\begin{minipage}[c]{0.75 \textwidth}
        \tt\begin{center} \vspace{#1} \end{center}\end{minipage}}\vspace{5 mm}\par}

        % Color selected text in red font.
        % \alert{text to color}
        \newcommand{\alert}[1]{{\color{red}#1}}

        % Color selected text in blue font.
        % \edited{text to color}
        \newcommand{\edited}[1]{{\color{blue}#1}}

        % Color selected text and add a "FIXME" note in the margin.
        % \fixme{text to color}
        \newcommand{\fixme}[1]{{\color{red}#1}
            \marginpar{\textsc{\color{red}fixme}}}

        % Color selected text (optional) and add a note in brackets.
        % \note[selected text]{comments}
        % \note{comments}
        \renewcommand{\note}[2][]{
            {\color{blue}#1 %
            [\textsc{note}:~#2]}
        }
        
        % Color selected text (optional) and add a note from someone.
        % \notefrom[selected text]{from}{comments}
        % \notefrom{from}{comments}
        \newcommand{\notefrom}[3][]{
            {\color{green!50!black}#1 %
            [\textsc{from #2}:~#3]}
        }
        
        % Color selected text (optional) and add a note to someone.
        % \noteto[selected text]{to}{comments}
        % \noteto{to}{comments}
        \newcommand{\noteto}[3][]{
            {\color{red}#1 %
            [\textsc{to #2}:~#3]}
        }

        % Color and Line Settings for Boxed Text
        \mdfsetup{
        % middlelinecolor=red,
        middlelinewidth=1pt,
        % linecolor=blue,
        % linewidth=1pt,
        backgroundcolor=orange!10!white,
        linecolor=orange!50!black,
        roundcorner=5pt}
        
        % Shortcut for referencing figures/tables:
        % Usage:  \figref{fig:name} --> Figure 1.
        \newcommand{\figref}[1]{\figurename~\ref{#1}}
        \newcommand{\tabref}[1]{\tablename~\ref{#1}}
    % ++++++++++++++++++++++++++++++++++++++++++++++++++++++++++++++++++++++++

    % SETUP TikZ +++++++++++++++++++++++++++++++++++++++++++++++++++++++++++++
        \usetikzlibrary{arrows,shapes,matrix}
        \usetikzlibrary{decorations.pathmorphing} 
        \usepgflibrary{plotmarks}
        \usetikzlibrary{patterns}  
        \usetikzlibrary{positioning} 
        \usetikzlibrary{snakes}  
        \tikzstyle{block}=[draw opacity=0.7,line width=1.4cm]
        
        % MORE STUFF TO ADD HERE?
    % ++++++++++++++++++++++++++++++++++++++++++++++++++++++++++++++++++++++++

    % SETUP BIBLIOGRAPHY +++++++++++++++++++++++++++++++++++++++++++++++++++++
    % [This section MUST be used if you have a bibliography.    ]
    % [Otherwise, leave this section commented out.        ]
    % \begin{comment}

        % FIXME -- EXPLAIN
        
        % Setup the Bibliography Style -- Select ONE of the following:
        % \usepackage{natbib}
        % \usepackage[sectionbib,square]{natbib}     %%% See natbib.pdf for explanation.
        % \usepackage[sectionbib,round]{natbib}
        \usepackage[square,numbers]{natbib}

        \bibliographystyle{plainnat}

        % Natbib setup for author-year style
        % \bibpunct has 1 optional and 6 mandatory arguments:
        %  [0.] The character preceding a post-note, default is a comma plus space. In redefining this character, 
        %     one must include a space if one is wanted. 
        %  1. the opening bracket symbol, default = (
        %  2. the closing bracket symbol, default = )
        %  3. the punctuation between multiple citations, default = ;
        %  4. the letter `n' for numerical style, or `s' for numerical superscript style, 
        %    any other letter for author-year, default = author-year;
        %  5. the punctuation that comes between the author names and the year
        %  6. the punctuation that comes between years or numbers when common author lists are suppressed (default = ,);

        % Natbib setup for author-year style
        \bibpunct[, ]{(}{)}{,}{a}{}{,}                % Use author names
        % \bibpunct[, ]{[}{]}{,}{n}{}{,}            % Use numbers
        
        \def\bibfont{\small}
        \def\bibsep{\smallskipamount}
        \def\bibhang{24pt}
        \def\newblock{\ }
        \def\BIBand{and}
    % \end{comment}
    % ++++++++++++++++++++++++++++++++++++++++++++++++++++++++++++++++++++++++

    % DOCUMENT INFO ++++++++++++++++++++++++++++++++++++++++++++++++++++++++++
        \newcommand{\docTitle}{}

        % List authors here, separated by \and 
        \newcommand{\docAuthor}{}
        % \newcommand{\docAuthor}{}

        \newcommand{\docAffil}{
            Department of Industrial \& Systems Engineering,\\%
            University at Buffalo, Buffalo, New York, USA%
        }

        \newcommand{\docAbstract}{}

        \newcommand{\docKeyword}{}

        % This date will appear under the title.
        \newcommand{\docDate}{\today}       % {} --> don't show a date.
            
        % This date will appear in the page header:
        \newcommand{\draftDate}{\today}    % {\today} --> draft, {} --> finalized (hidden)
    
        % The image files should be saved here:
        \graphicspath{ {../../image/} }
    % ++++++++++++++++++++++++++++++++++++++++++++++++++++++++++++++++++++++++

    % DEFINE HEADER ++++++++++++++++++++++++++++++++++++++++++++++++++++++++++
        \usepackage{fancyhdr}
        \pagestyle{fancy}
        \ifthenelse{\equal{\draftDate}{}}
            {
                % This is the final version...remove the date from the header
                \chead{}
            }
            {
                % This is a working draft...include the date in the header
                % \chead{\color{red}DRAFT -- Updated \draftDate~at~\currenttime}
            }
        \lhead{}    % no left/right header content
        \rhead{}
        %\cfoot{}
        %\lfoot{}
        %\rfoot{}
        \renewcommand{\headrulewidth}{0pt}
        \renewcommand{\footrulewidth}{0pt}
        %\fancyfoot{}
    % ++++++++++++++++++++++++++++++++++++++++++++++++++++++++++++++++++++++++
    
    % DEFINE PROGRAMMING FORMAT ++++++++++++++++++++++++++++++++++++++++++++++
    \lstset{language=Python}          % Set your language (you can change the language for each code-block optionally)

    \definecolor{mygreen}{rgb}{0,0.6,0}
    \definecolor{mygray}{rgb}{0.5,0.5,0.5}
    \definecolor{mymauve}{rgb}{0.58,0,0.82}

    \lstset{ %
      backgroundcolor=\color{gray!05!white},   % choose the background color; you must add \usepackage{color} or \usepackage{xcolor}; should come as last argument
      basicstyle=\ttfamily,        % the size of the fonts that are used for the code
      breakatwhitespace=false,         % sets if automatic breaks should only happen at whitespace
      breaklines=true,                 % sets automatic line breaking
      captionpos=t,                    % sets the caption-position to bottom
      commentstyle=\color{black},    % comment style
      deletekeywords={...},            % if you want to delete keywords from the given language
      escapeinside={\%*}{*)},          % if you want to add LaTeX within your code
      extendedchars=true,              % lets you use non-ASCII characters; for 8-bits encodings only, does not work with UTF-8
      frame=single,                       % adds a frame around the code
      keepspaces=true,                 % keeps spaces in text, useful for keeping indentation of code (possibly needs columns=flexible)
      % keywordstyle=\color{blue},       % keyword style
      language=Python,                 % the language of the code
      morekeywords={*,...},           % if you want to add more keywords to the set
      numbers=none,                    % where to put the line-numbers; possible values are (none, left, right)
      numbersep=5pt,                   % how far the line-numbers are from the code
      % numberstyle=\tiny\color{mygray}, % the style that is used for the line-numbers
      rulecolor=\color{black},         % if not set, the frame-color may be changed on line-breaks within not-black text (e.g. comments (green here))
      showspaces=false,                % show spaces everywhere adding particular underscores; it overrides 'showstringspaces'
      showstringspaces=false,          % underline spaces within strings only
      showtabs=false,                  % show tabs within strings adding particular underscores
      stepnumber=1,                    % the step between two line-numbers. If it's 1, each line will be numbered
      % stringstyle=\color{mymauve},     % string literal style
      tabsize=4,                       % sets default tabsize to 2 spaces
      % title=\lstname,                   % show the filename of files included with \lstinputlisting; also try caption instead of title
      xleftmargin=35pt,
      xrightmargin=15pt, 
      aboveskip=0pt,
      belowskip=5pt
    }
    % ++++++++++++++++++++++++++++++++++++++++++++++++++++++++++++++++++++++++

    \newcommand{\titleSec}{
        % See https://tex.stackexchange.com/questions/216098/redefine-maketitle
        \begin{center}
        % \let \footnote \thanks
        {\Large \textbf{\docTitle} \par}

        % Authors?
        % Comment these lines out if you want to hide authors
        \vskip 1.0em%
        \lineskip .5em%
        \begin{tabular}[t]{c}
            \docAuthor
        \end{tabular}\par%

        % Affiliation?
        % Comment these lines out if you want to hide affiliation info
        \vskip 1.0em%
        {\small \docAffil \par}

        % Displayed date?
        % Comment these lines out if you want to hide the date
        %\vskip 1.0em%
        %{\small \docDate \par}  

        \end{center}
        \par
        \vskip 1.5em

        % \begin{abstract}
        %     \docAbstract
        % \end{abstract}

        % \begin{keywords}
        %     \docKeyword
        % \end{keywords}

        % This is version \texttt{\templateVersion} of this template.
        % Visit \templatesURL for the latest versions.
    }
\usepackage{makecell}

\usetikzlibrary{shapes.geometric, arrows}
    \tikzstyle{startstop} = [rectangle, rounded corners, minimum width=3cm, minimum height=1cm,text centered, draw=black]
    \tikzstyle{io} = [trapezium, trapezium left angle=70, trapezium right angle=110, minimum width=3cm, minimum height=1cm, text centered, draw=black]
    \tikzstyle{process} = [rectangle, minimum width=2cm, minimum height=1cm, text centered, draw=black, inner sep=0.1cm]
    \tikzstyle{decision} = [diamond, minimum width=2cm, minimum height=0cm, text centered, draw=black, inner sep=0cm]
    \tikzstyle{arrow} = [thick,->,>=stealth]
    \tikzstyle{branchnode} = [circle, minimum size = 1cm, text centered, draw=black, inner sep=0.1cm]

\renewcommand{\docTitle}{Lecture Note - Traveling Salesman Problem and The Held-Karp Lower Bound}
\renewcommand{\docAuthor}{Lan Peng, Ph.D.}
\renewcommand{\docAffil}{School of Management, Shanghai University, Shanghai, China}
\begin{document}
    \titleSec

    \begin{center}
        \textit{``O never go back.''}
    \end{center}

    \section{The Traveling Salesman Problem}
        In this section, we are going to compare between different formulations of the Traveling Salesman Problem (TSP). Generally speaking, let $G = (V, A)$ be a graph where $V$ is a set of $n$ vertices, and $A$ is a set of arcs (or edges). Let $C = c_{ij}$ be a cost (distance) matrix associated with $A$. The TSP consists of determining a minimum cost (distance) Hamiltonian circle (or cycle) that visits each vertex once and only once. If for all $i, j \in V, c_{ij} = c_{ji}$, then the TSP is symmetrical, otherwise is asymmetrical.

        Define the decision variable $x_{ij}$ as the following
        \begin{equation}
            x_{ij} = \begin{cases}
                1, &\text{if goes from } i \text{ to } j\\ 
                0, & \text{otherwise}
            \end{cases}, \quad (i, j) \in A
        \end{equation}

        The objective function will be
        \begin{equation}
            \min \quad \sum_{(i, j)\in A} c_{ij}x_{ij}
        \end{equation}

        \subsection{Dantzig-Fulkerson-Johnson (DFJ) Formulation}
            The first famous formulations for TSP is the \textbf{Dantzig-Fulkerson-Johnson (DFJ) formulation}:
            \begin{align}
                \sum_{j \in V, (i,j)\in A} x_{ij} & = 1, \quad \forall i \in V \label{TSP:con:degree1}\\
                \sum_{i \in V, (i,j)\in A} x_{ij} & = 1, \quad \forall j \in V \label{TSP:con:degree2}\\
                \sum_{j\notin S, i\in S, (i,j)\in A} x_{ij} & \ge 1, \quad \forall S \subset V, 2\le |S| \le n-1 \label{TSP:con:DFJSubtour1}
            \end{align}

            In the formulation, constraints (\ref{TSP:con:degree1}) and constraints (\ref{TSP:con:degree2}) are degree constraints, which specify that every vertex is entered exactly once. Constraints (\ref{TSP:con:DFJSubtour1}) is the sub-tour constraints, they prohibit the formation of sub-tours. $S$ is a non-empty subset of $V$, and has at least 2 vertices. (\ref{TSP:con:DFJSubtour1}) can be replaced by
            \begin{equation}
                \sum_{i, j \in S, (i, j) \in A} x_{ij} \le |S| - 1, \quad \forall S \subset V, 2\le |S| \le n-1\label{TSP:con:DFJSubtour2}
            \end{equation}

            If we list all sub-tour constraints in DFJ, there will be $O(2^n)$ constraints and $O(n^2)$ binary variables. The exponential number of constraints makes it impractical to solve directly. Instead, lazy constraints are usually implemented for the sub-tour elimination constraints (\ref{TSP:con:DFJSubtour1}) or (\ref{TSP:con:DFJSubtour2}).

             In the graph $G=(N, A)$, let $\bar{G}=(N, \bar{A})$ be the connected components of graph, where
            \begin{equation*}
                \bar{G}=(G, \bar{A}), \bar{A}=\{(i, j) \in A | \bar{x_{ij}}=1\}
            \end{equation*}
            
            denote
            
            \begin{equation*}
                \bar{FS}(i) = \{(i,j)\in \bar{A}\} 
            \end{equation*}

            The following algorithm describes an algorithm to find subtours:
            \begin{algorithm}[h!]
                \caption{Sub-tour Searching Algorithm}
                \begin{algorithmic}[1]
                    \State $K = \emptyset$
                    \State $d_i = 0, \forall i \in N$
                    \For {$i\in N$}
                        \State $C = \emptyset$
                        \State $Q = \emptyset$
                        \If {$d_i == 0$}
                            \State $d_i = 1$
                            \State $C = C\cup \{i\}$
                            \State Q.append(i)
                            \While {$Q\ne \emptyset$}
                                \State v = Q.pop()
                                \For {$u \in \bar{FS}(v)$}
                                    \If {$d_u == 0$}
                                        \State $d_u = 1$
                                        \State $C = C \cup \{u\}$
                                        \State Q.append(u)
                                    \EndIf
                                \EndFor
                            \EndWhile
                        \EndIf
                        \State $K=K\cup C$
                    \EndFor
                \end{algorithmic}
            \end{algorithm}

            We can add those sub-tour constraints into the formulation on the fly in a Branch-and-cut manner.

        \subsection{Miller-Tucker-Zemlin (MTZ) Formulation}
            We can also formulate TSP using sequential formulations, namely, \textbf{Miller-Tucker-Zemlin (MTZ) formulation}. In the MTZ formulation, the degree constraints (\ref{TSP:con:degree1}) and (\ref{TSP:con:degree2}) are the same as in DFJ formulation.

            Define a new set of integer decision variables $u_i$, $u_i$ defined as the sequence in which node $i$ is visited, $u_1 = 1$.

            The sub-tour constraints (\ref{TSP:con:DFJSubtour1}) or (\ref{TSP:con:DFJSubtour2}) are replaced by the following:
            \begin{align}
                u_i - u_j + (n - 1) x_{ij} &\le n - 2, \quad i, j = 2, \cdots, n \in V, (i, j) \in A \label{TSP:con:MTZ1}\\
                1 & \le u_i \le n - 1, \quad i \in 2, \cdots, n \in V \label{TSP:con:MTZ2}
            \end{align}

            In MTZ formulation, there are $O(n^2)$ constraints, $O(n^2)$ binary variables, and $O(n)$ continuous variables.

        \subsection{Flow Based Formulations}
            In this section, flow based formulations are discussed, which includes \textbf{Single Commodity Flow}, \textbf{Two Commodity Flow} and \textbf{Multi-Commodity Flow}. In these formulations, continuous variables are introduced to represent the flow on the arcs.

            In Single Commodity Flow formulation, define $y_{ij}$ as the flow in an arc $(i, j) \in A$. Degree constraints (\ref{TSP:con:degree1}) and (\ref{TSP:con:degree2}) are retained. The following constraints are introduced:
            \begin{align}
                y_{ij} & \le (n - 1) x_{ij}, \quad \forall i, j \in V, (i, j) \in A \label{TSP:con:SCFMaxFlow}\\
                \sum_{j \in V, (1, j) \in A} y_{1j} & = n - 1 \label{TSP:con:SCFInitFlow} \\
                \sum_{i \in V, (i, j) \in A} y_{ij} - \sum_{k \in V, (j, k) \in A} y_{jk} &= 1, \quad \forall j \in V \setminus \{1\} \label{TSP:con:SCFFlowBalance}
            \end{align}

            Constraints (\ref{TSP:con:SCFMaxFlow}) can be tighten by the following:
            \begin{align}
                y_{ij} &\le (n - 1) x_{ij}, \quad i = 1, j \in V \setminus \{1\}, (i, j) \in A \label{TSP:con:SCMMaxFlow1} \\
                y_{ij} &\le (n - 2) x_{ij}, \quad i, j \in V \setminus \{1\}, (i, j) \in A \label{TSP:con:SCMMaxFlow2}
            \end{align}

            In SCM formulation, there are $O(n^2)$ constraints, $O(n^2)$ binary variables and $O(n^2)$ continuous variables.

            In Two Commodity Flow formulation, define $y_{ij}$ as the flow in an arc $(i, j) \in A$, for commodity type 1, and define $z_{ij}$ as the flow in an arc $(i, j) \in A$, for commodity type 2.

            Besides degree constraints, the other constraints are as following
            \begin{align}
                y_{ij} + z_{ij} &= (n - 1) x_{ij}, \quad \forall i, j \in V, (i, j) \in A \label{TSP:con:TCMFlowExist} \\
                \sum_{j \in V \setminus \{1\}} (y_{1j} - y_{j1}) &= n - 1, \quad (1, j) \in A \label{TSP:con:TCMInitFlowY}\\
                \sum_{j \in V} (y_{ij} - y_{ji}) & = 1, \quad  \forall i \in V \setminus \{1\}, (i, j) \in A \label{TSP:con:TCMFlowBalanceY}\\
                \sum_{j \in V \setminus \{1\}} (z_{1j} - z_{j1}) &= 1 - n, \quad (1, j) \in A \label{TSP:con:TCMInitFlowZ}\\
                \sum_{j \in V} (z_{ij} - z_{ji}) & = -1, \quad  \forall i \in V \setminus \{1\}, (i, j) \in A \label{TSP:con:TCMFlowBalanceZ}\\
                \sum_{j \in V} (y_{ij} + z_{ij}) &= n - 1, \quad \forall i \in V \label{TSP:con:TCMFlowOnArc}
            \end{align}

            In TCM formulation, constraints (\ref{TSP:con:TCMFlowExist}) only allow flow in an arc if present. Constraints (\ref{TSP:con:TCMInitFlowY}) and (\ref{TSP:con:TCMFlowBalanceY}) forces $(n - 1)$ units of commodity type 1 to flow in at node 1 and 1 unit to flow out at every other nodes. Constraints (\ref{TSP:con:TCMInitFlowZ}) and (\ref{TSP:con:TCMFlowBalanceZ}) are similar, those forces $(n - 1)$ units of commodity type 2 to flow out at node 1 and 1 unit to flow in at every other nodes. Constraints (\ref{TSP:con:TCMFlowOnArc}) forces exactly $(n - 1)$ units of combined commodity in each arc.

            In TCM formulation, there are $O(n^2)$ constraints, $O(n^2)$ binary variables and $O(n^2)$ continuous variables.

            The SCM and the TCM can be generalized into \textbf{Multi-Commodity Flow formulation}. As usual, degree constraints are retained. The following continuous variables are introduced. Define $y_{ij}^k$ as the flow of commodity type $k$ in arc $(i, j) \in A$.

            The other constraints are
            \begin{align}
                y_{ij}^k &\le x_{ij}, \quad \forall i, j, k \in N, k \neq 1 \label{TSP:con:MCMFlowExist}\\
                \sum_{i \in V} y_{1i}^k &= 1, \quad \forall k \in V \setminus \{1\} \label{TSP:con:MCMInitIn}\\
                \sum_{i \in V} y_{i1}^k &= 0, \quad \forall k \in V \setminus \{1\} \label{TSP:con:MCMInitOut}\\
                \sum_{i \in V} y_{ik}^k &= 1, \quad \forall k \in V \setminus \{1\} \label{TSP:con:MCMElseOut}\\
                \sum_{j \in V} y_{kj}^k &= 0, \quad \forall k \in V \setminus \{1\} \label{TSP:con:MCMElseIn}\\
                \sum_{i \in V} y_{ij}^k - \sum_{i \in V} y_{ji}^k &= 0, \quad \forall j, k \in V \setminus \{1\}, j \neq k \label{TSP:con:MCMBalance}
            \end{align}
            Constraints (\ref{TSP:con:MCMFlowExist}) only allow flow in an arc which is present. Constraints (\ref{TSP:con:MCMInitIn}) forces exactly one unit of each type of commodity to flow in at node 1. Constraints (\ref{TSP:con:MCMInitOut}) prevent any commodity flow out at node 1.Constraints (\ref{TSP:con:MCMElseOut}), and Constraints (\ref{TSP:con:MCMElseIn}), forces exactly one unit of type $k$ commodity to flow out, and in, at every node except node 1. Constraints (\ref{TSP:con:MCMBalance}) forces balance of all types of commodities at every node except node 1.

            This formulation has $O(n^3)$ constraints, $O(n^2)$ binary variables, and $O(n^3)$ continuous variables.

        \subsection{Shortest Path Formulation}
            In this section, we are going to introduce another form of formulation with different definition of decision variable and objective function.

            Assuming for a completed graph $G = (V, A)$. Define $x_{ij}^t$ as the following
            \begin{equation}
                x_{ij}^t = \begin{cases}
                                1, \quad \text{If path crosses arc } (i, t) \text{ and } (j, t + 1) \\
                                0, \quad \text{Otherwise}
                            \end{cases}, \quad i \in V, j \in V \setminus \{i\}, t = 1, \cdots, n
            \end{equation}

            \begin{figure}[!h]
                \centering
                \includegraphics[width=0.6\textwidth]{../../image/timeStaged.png}
                \caption{Time-staged graph}
                \label{fig:timeStaged}
            \end{figure}

            The objective function will be
            \begin{equation}
                \min \quad \sum_{i \in V}\sum_{j \in V\setminus \{i\}} c_{ij} \sum_{t = 1}^n x_{ij}^t
            \end{equation}

            The constraints are as following
            \begin{align}
                \sum_{j \in V \setminus \{1\}} x_{1j}^1 &= 1 \label{TSP:con:SPFStart}\\
                \sum_{j \in V \setminus \{1, i\}} x_{ij}^2 - x_{1i}^1 &= 0, \quad \forall i \in V \setminus \{1\} \label{TSP:con:SPFFirstLayer}\\
                \sum_{j \in V \setminus \{1, i\}} x_{ij}^t - \sum_{j \in V \setminus \{1, i\}} x_{ji}^{t - 1} &= 0, \quad \forall i \in V \setminus \{1\}, t \in \{2, \dots, n - 1\} \label{TSP:con:SPFTthLayer}\\
                x_{i1}^n - \sum_{j \in V \setminus \{1, i\}} x_{ji}^{n - 1} &= 0, \quad \forall i \in V \setminus \{1\} \label{TSP:con:SPFLastLayer}\\
                \sum_{i \in V \setminus \{1\}} x_{i1}^n &= 1 \label{TSP:con:SPFEnd}\\
                \sum_{t = 2}^{n - 1}\sum_{j \in V \setminus \{1, i\}} x_{ij}^t + x_{i1}^n & \le 1, \quad \forall i \in V \setminus \{1\} \label{TSP:con:SPFSameType1}
            \end{align}

            Notice that constraint (\ref{TSP:con:SPFSameType1}) can be replaced by
            \begin{equation}
                x_{1i}^1 + \sum_{t = 2}^{n - 1}\sum_{j \in V \setminus \{1, i\}} x_{ji}^t \le 1, \quad \forall i \in V \setminus \{1\} \label{TSP:con:SPFSameType2}
            \end{equation}

        \subsection{Quadratic Formulation (QAP)}
            In this section, we are going to go over a TSP formulation are super bad. However, it still has some value for further study.

            The idea is to transform TSP into an assignment problem. Assuming we have $n$ boxes, which represents $n$ steps in the path. Define $x_{ij}$ as 
            \begin{equation}
                x_{ij} = \begin{cases}
                            1, \quad \text{Vertex $i$ is assigned to box $j$}\\
                            0, \quad \text{Otherwise}
                        \end{cases}
            \end{equation}

            The constraints are simple as an assignment problem as following
            \begin{align}
                \sum_{j = 1}^n x_{ij} &= 1, \quad \forall i \in V\\
                \sum_{i \in V}^n x_{ij} &= 1, \quad j = 1, \dots, n
            \end{align}

            However, the tricky part is in the objective function
            \begin{equation}
                \min \quad \sum_{i \in V} \sum_{j \in V \setminus \{i\}} \sum_{k = 1}^{n - 1} c_{ij} x_{ik} x_{j, k + 1} + \sum_{i \in V} \sum_{j \in V \setminus \{i\}} c_{ij}x_{in}x_{j1}
            \end{equation}

            Notice that the objective function is not linear function, with the multiplications of decision variables. Now we are going to linearize them. The linearized version is as following

            \begin{align}
                \min \quad & \sum_{i \in V} \sum_{j \in V \setminus \{i\}} \sum_{k = 1}^{n - 1} c_{ij} w_{ij}^k + \sum_{i \in V} \sum_{j \in V \setminus \{i\}} c_{ij}w_{ij}^n\\
                \text{s.t.} \quad & \sum_{j = 1}^n x_{ij} = 1, \quad \forall i \in V\\
                                  & \sum_{i \in V}^n x_{ij} = 1, \quad j = 1, \dots, n\\
                                  & w_{ij}^k \ge x_{ik} + x_{j, k + 1} - 1, \quad i \in V,  j \in V \setminus \{i\}, k = 1, \cdots, n - 1\\
                                  & w_{ij}^k \ge x_{ik} + x_{j1} - 1, \quad i \in V, j \in V \setminus \{i\}, k = n \\
                                  & w_{ij}^k \in \{0, 1\}, \quad i \in V, j \in V \setminus \{i\}, k = 1, \dots, n\\
                                  & x_{ij} \in \{0, 1\}, \quad i \in V, j \in V \setminus \{i\}
            \end{align}

            We can prove that this is very very bad. The optimal solution of the LP Relaxation for the QAP formulation is as follows

            \begin{align}
                x_{ij} &= \frac{1}{n}, \quad \forall i, j \in V\\
                w_{ij}^k &= \frac{2}{n} - 1, \quad \forall i \in V, j \in V \setminus \{i\}, k = 1, 2, \ldots, n
            \end{align}

            The solution indicates that all decision variables are symmetric in the LP Relaxation, thus, it did not provide any information for branching. In fact, such formulation will search all $O(2^n)$ branches and the lower bound will be difficult to converge.
    
    \section{The Held and Karp Lower Bound}
        In this section, we will solve the Dantzig-Fulkerson-Johnson formulation using Lagrangian Relaxation. Before finally converge, the LR finds an infeasible solution as lower bound. The bound found by this method is also known as Held \& Karp Bound.
        
        \subsection{Minimum Spanning Tree}
            We first introduce the concept of minimum spanning tree by an example. A company wants to build a communication network for their offices. For a link between office $v$ and office $w$, there is a cost $c_{vw}$. If an office is connected to another office, then they are connected to with all its neighbors. Company wants to minimize the cost of communication networks.

            \begin{definition}[Spanning Tree]
                A subgraph T of G is a \textbf{spanning tree} if it is spanning ($V(T)=V(G)$) and it is a tree.
            \end{definition}

            \begin{definition}[Minimum Spanning Tree]
                Given a connected graph graph $G$, and a cost $C_e, \forall e\in E$, find a minimum cost spanning tree of $G$
            \end{definition}

            The follow are two algorithms that finds the minimum spanning tree on given graph $G$.

            \begin{algorithm}
                \caption{Kroskal's Algorithm, $O(m \log m)$}
                \begin{algorithmic}
                    \Require A connected graph
                    \Ensure A minimum spanning tree
                    \State Keep a spanning forest $H=(V, F)$ of $G$, with $F=\emptyset$
                    \While {$|F| < |V| - 1$}
                        \State add to $F$ a least-cost edge $e\notin F$ such that $H$ remains a forest.
                    \EndWhile
                \end{algorithmic}
            \end{algorithm}

            \begin{algorithm}
                \caption{Prim's Algorithm, $O(nm)$}
                \begin{algorithmic}
                    \Require A connected graph
                    \Ensure A minimum spanning tree
                    \State Keep $H = (V(H), T)$ with $V(H) = \{v\}$, where $r\in V(G)$ and $T=\emptyset$
                    \While {$|V(T)| < |V|$}
                        \State Add to $T$ a least-cost edge $e \notin T$ such that $H$ remains a tree.
                    \EndWhile
                \end{algorithmic}
            \end{algorithm}

            \begin{itemize}
                \item Kroskal start with a forest that contains all vertices, Prim start with a tree that only contain one vertex.
                \item Kroskal cannot guarantee every step it is a tree but can guarantee it is spanning, Prim can guarantee every step it is a tree but cannot guarantee spanning.
            \end{itemize}

        \subsection{1-Tree}
            We first introduce an intuitive lower bound for TSP, which is the Minimum Spanning Tree Lower Bound. As we known, an optimal solution for the TSP is a Hamilton Cycle which enumerated all vertices on the graph. The length of such Hamilton cycle is denoted as $TSP^*$. If we randomly remove one of the edges, e.g., $\{vw\}$, then, the cycle becomes a path, denoted by $TSP - \{vw\}$. In this case, the degree of vertices $v$ and $w$ reduce to 1 while all the other vertices remains the same as 2. By definition, such path $TSP - \{vw\}$ is a spanning tree on graph $G$. Therefore, the minimum spanning tree of graph $G$ defines a lower bound.

            \begin{equation*}
                MST \le spanning tree = TSP - \{vw\} < TSP^*
            \end{equation*}

            We can further improve the lower bound by introducing 1-tree. A spanning tree is a tree with no cycle, if the graph has 1 cycle, it is called 1-tree. Notice that a Hamilton Cycle has only 1 cycle, thus, the Hamilton Cycle is an 1-tree as well. Naturally, the length of all edges of any 1-tree is larger than the length of all edges of the minimum spanning tree. The lower bound of TSP is further improved as follows

            \begin{equation*}
                MST < M1T \le 1-tree \le TSP^*
            \end{equation*}

            The following algorithm defines an 1-Tree on graph $G$ corresponding to vertex $v$.

            \begin{algorithm}
                \caption{1-Tree}
                \begin{algorithmic}
                    \Require A connected graph
                    \Ensure A minimum 1-Tree
                    \State Remove $v$ from the graph
                    \State Use Kroskal's algorithm or Prim's Algorithm to find the minimum spanning tree of $G \setminus \{v\}$
                    \State Find the shortest edge induced by $v$ to the rest of graph $G \setminus \{v\}$, denoted by $vu$ and $vw$, add them to the MST
                    \State \Return 1-Tree
                \end{algorithmic}
            \end{algorithm}

            \begin{figure}[!htp]
                \centering                
                \subfloat[Remove $v$]{\includegraphics[width=0.27\textwidth]{../../image/1Tree1.png}}\quad
                \subfloat[Find MST]{\includegraphics[width=0.27\textwidth]{../../image/1Tree2.png}}\quad
                \subfloat[Add back $v$]{\includegraphics[width=0.27\textwidth]{../../image/1Tree3.png}}
                \caption{Steps in finding minimum 1-Tree}
                \label{fig:1tree}
            \end{figure}

            A M1T is a good lower bound, however, we can still further improve the lower bound by finding 1-trees.

        \subsection{Held and Karp Lower Bound and Lagrangian Relaxation}
            Notice that in the minimum 1-tree example, vertices have different degrees, some are of degree 1, 2, 3, or more. However, the ``optimal'' 1-tree that we look for, is an 1-tree such that all the vertices are of degree 2. By intuition, if an 1-tree has fewer ``branches'', it's closer to be ``optimal''. That is actually the principle of the Held and Karp Lower Bound.

            We first look at the DFJ formulation for the TSP.

            \begin{align}
                \min \quad &\sum_{e \in E} \tau_e x_e \label{lr:obj}\\
                \text{s.t.} \quad & \sum_{e \in \delta(i)} x_e = 2, \quad \forall i \in 1, 2, \ldots, n\label{lr:deg}\\
                & \sum_{e \in E(S)} x_e \le |S| - 1, \quad \forall S \subset V, 2 \le |S| \le n - 1 \label{lr:subtour}\\
                & x_e \in \{0, 1\}, \quad \forall e \in E
            \end{align}

            In the DFJ model, the objective is to sum up all the cost of the edges that are chosen on the TSP path. Constraints (\ref{lr:deg}) is a flow balancing constraint, the index $e\in \delta(i)$ represents all the edges that are induced by vertex $i$. Constraints (\ref{lr:subtour}) is the sub-tour constraint.

            Replace the Constraints (\ref{lr:deg}) by the following
            \begin{align}
                \sum_{e \in \delta(i)} x_e &= 2, \quad \forall i \in 1, 2, \ldots, n - 1\\
                \sum_{e \in E} x_e &= n
            \end{align}

            We can reformulate the DFJ formulation as follows:

            \begin{align}
                \min \quad &\sum_{e \in E} \tau_e x_e\\
                \text{s.t.} \quad & \sum_{e \in \delta(i)} x_e = 2, \quad \forall i \in 1, 2, \ldots, n - 1\label{lr:ndeg}\\
                &\sum_{e \in E} x_e = n \label{lr:cycle}\\
                & \sum_{e \in E(S)} x_e \le |S| - 1, \quad \forall S \subset V, 2 \le |S| \le n - 1 \label{lr:nsubtour}\\
                & x_e \in \{0, 1\}, \quad \forall e \in E
            \end{align}

            Look closely, Constraints (\ref{lr:cycle}) means, the number of edges in the subgraph should be the same as the number of vertices, so the subgraph has 1 cycle. The Constraints (\ref{lr:nsubtour}) guarantee that the graph is connected. Which means, Constraints (\ref{lr:cycle}) and (\ref{lr:nsubtour}) finds an 1-tree.

            Until now, we can move the Constraints (\ref{lr:ndeg}) to the objective function, and use the Lagrangian Relaxation to solve the problem

            \begin{align}
                z(\mathbf{u}) = \min \quad & \sum_{e \in E} \tau_e x_e + \sum_{i = 1}^{n - 1}u_i(2 - \sum_{e \in \delta(i)} x_e)\\
                \text{s.t.} \quad & \mathbf{x} \text{ defines an 1-tree with vertice } n
            \end{align}

            For the vertex $i$, define $u_i$, the formulation can be further rewritten as

            \begin{align}
                z(\mathbf{u}) = \min \quad & 2\sum_{i = 1}^{n - 1} u_i + \sum_{e \in \delta(i)} (\tau_e - u_{e^+} - u_{e^-})x_e\\
                \text{s.t.} \quad & \mathbf{x} \text{ defines an 1-tree with vertice } n
            \end{align}

        \subsection{Subgradient Descendant Method}
            Notice that in the previous section, we derived a function $z(\mathbf{u})$ of Lagrangian scalar $u_i$ as the lower bound of the TSP. The solution space of $\mathbf{u}$ is a convex space, in this section, we will search the $\max_{\mathbf{u}} z(\mathbf{u})$ by subgradient descendant method.

            \begin{algorithm}
                \caption{Subgradient descendant method for Held-Karp Lower Bound}
                \begin{algorithmic}
                    \Require A connected graph
                    \Ensure A lower bound of TSP
                    \State Initialize, for each vertex $i$, let $u_i = 0$, $d_i = \emptyset$. Define lower bound $L \gets 0$, $L^\prime \gets 0$
                    \While $|L - L^\prime| \ge \epsilon$
                        \State Update weights of edges $\tau_{ij} = \tau_{ij} - u_i - u_j$
                        \State Find the M1T on the graph $G$ with edges updated. Let $D$ be the summation of the length of all edges, let $d_i$ be the degree of vertex $i$ on the M1T.
                        \State Update $u_i \gets u_i + \lambda_i \frac{UB - D}{\sum_i (d_i - 2)^2}$
                        \State Update $L^\prime \gets L$
                        \State Update $L \gets D + 2 \sum_i u_i$
                    \EndWhile
                    \State \Return L
                \end{algorithmic}
            \end{algorithm}

            In the algorithm, different descendant function may be applied, for example, an easier way for implementation is to update
            \begin{equation*}
                u_i \gets u_i + (d_i - 2) \rho^k
            \end{equation*}

    % \bibliography{literature}
\end{document}